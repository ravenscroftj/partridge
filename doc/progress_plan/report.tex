% Outline Project Specification Report

%The document is a report
\documentclass[12pt,a4paper]{article}

%define horizontal rule
\newcommand{\HRule}{\rule{\linewidth}{0.5mm}}

\usepackage{fullpage}

%use the listings package
\usepackage{listings}
%use the English language
\usepackage[english]{babel}
%use graphics
\usepackage{graphicx}
%use wrap figures
\usepackage{wrapfig}
%geometry stuffs
\usepackage{lscape}
%use natbib bibliography package
%\usepackage[numbers]{natbib}
%use harvard bibliography package
%\usepackage{harvard}	
%use captions
\usepackage{caption}
%use multi-row tables
\usepackage{multirow}
\usepackage{url}

\begin{document}

%use harvard citations
%\citationstyle{agsm}

%include the title page
\begin{titlepage}
 
\begin{center}
 
 
% Upper part of the page
\includegraphics[width=0.20\textwidth]{./Gerald_G_Man_in_Suit.png}\\[1cm]
 
\textsc{\LARGE Aberystwyth University}\\[1.5cm]
 
\textsc{\LARGE Industrial Year Report}\\[0.5cm]
 
 
% Title
\HRule \\[0.4cm]
{ \huge \bfseries Blue Harvest - My Year at IBM }\\[0.4cm]
 
\HRule \\[1.5cm]

 % Author and student ID
\begin{minipage}{0.4\textwidth}
\begin{flushleft} \large
\emph{Author:}\\
James \textsc{Ravenscroft}
\end{flushleft}
\end{minipage}
\begin{minipage}{0.4\textwidth}
\begin{flushright} \large
\emph{Student ID:} \\
090407039
\end{flushright}
\end{minipage}

\vfill
 
% Bottom of the page
{\large \today}
 
\end{center}
 
\end{titlepage}



%some definitions for paragraph layout stuff
\setlength{\parindent}{0pt}
\setlength{\parskip}{1.5ex plus 0.5ex minus 0.2ex}

\tableofcontents

\pagebreak

\section{Project Summary}
Partridge is a web-based tool designed to assist research and knowledge
acquisition in the domain of scientific papers.

With the advent of the 'Digital Age' and the ability of computers to copy and
reproduce information for a negligible cost, the amount of information
available to researchers is increasing drastically.  Bj\"{o}rk's 2009 study
suggests that approximately 1.4 Million papers were published in the year 2006
alone\cite{bjork2009}. Moreover, the growing popularity of Open Access
publishing (making papers available for free online\cite{Suber2012}) across
most scientific disciplines is providing researchers with an even larger volume
of information to be processed. It logically follows that as the amount of
general information grows, the proportion of relevant material becomes harder
to find without the assistance of complicated information filtering and retrieval
systems.

Partridge aims to autonomously process as many scientific papers as
possible to facilitate intelligent filtering and recommendations to
researchers, who would otherwise be required to manually browse thousands of
online documents. This should reduce the amount of information that the user is
required to process themselves, thereby speeding up the research process.

\section{Current Progress}

\section{Planning}

\renewcommand{\refname}{Initial Bibliography} 
\bibliographystyle{plain}
\bibliography{report}

\end{document}
