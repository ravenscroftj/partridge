% Outline Project Specification Report

%The document is a report
\documentclass[12pt,a4paper]{article}

%define horizontal rule
\newcommand{\HRule}{\rule{\linewidth}{0.5mm}}

\usepackage{fullpage}

%use the listings package
\usepackage{listings}
%use the English language
\usepackage[english]{babel}
%use graphics
\usepackage{graphicx}
%use wrap figures
\usepackage{wrapfig}
%geometry stuffs
\usepackage{lscape}
%use natbib bibliography package
%\usepackage[numbers]{natbib}
%use harvard bibliography package
%\usepackage{harvard}	
%use captions
\usepackage{caption}
%use multi-row tables
\usepackage{multirow}
\usepackage{url}

\begin{document}

%use harvard citations
%\citationstyle{agsm}

%include the title page
\begin{titlepage}
 
\begin{center}
 
 
% Upper part of the page
\includegraphics[width=0.20\textwidth]{./Gerald_G_Man_in_Suit.png}\\[1cm]
 
\textsc{\LARGE Aberystwyth University}\\[1.5cm]
 
\textsc{\LARGE Industrial Year Report}\\[0.5cm]
 
 
% Title
\HRule \\[0.4cm]
{ \huge \bfseries Blue Harvest - My Year at IBM }\\[0.4cm]
 
\HRule \\[1.5cm]

 % Author and student ID
\begin{minipage}{0.4\textwidth}
\begin{flushleft} \large
\emph{Author:}\\
James \textsc{Ravenscroft}
\end{flushleft}
\end{minipage}
\begin{minipage}{0.4\textwidth}
\begin{flushright} \large
\emph{Student ID:} \\
090407039
\end{flushright}
\end{minipage}

\vfill
 
% Bottom of the page
{\large \today}
 
\end{center}
 
\end{titlepage}



%some definitions for paragraph layout stuff
\setlength{\parindent}{0pt}
\setlength{\parskip}{1.5ex plus 0.5ex minus 0.2ex}

\tableofcontents

\pagebreak

\section{Project Summary}

\subsection{Introduction}
Partridge is a web-based tool designed to assist information processing and knowledge
acquisition in the domain of scientific research.

Since the advent of the 'Digital Age' and the ability of computers to copy and
reproduce information for a negligible cost, the amount of information
available to researchers has been increasing drastically.  A 2009 study by B-C
Bj\"{o}rk suggests that approximately 1.4 Million papers were published in the
year 2006 alone\cite{bjork2009}. Moreover, the growing popularity of Open Access
publishing (making papers available for free online\cite{Suber2012}) across
most scientific disciplines is providing researchers with an even larger volume
of information to be processed. It logically follows that as the amount of
general information grows, the proportion of relevant material becomes harder
to find without the assistance of complicated information filtering and retrieval
systems.

Partridge aims to autonomously process as many scientific papers as possible to
facilitate researchers who would otherwise be required to manually read each
paper themselves. This should reduce the amount of information that the user is
required to process themselves, thereby speeding up the research process.
Partridge will achieve this through the use of several existing techniques in
the field of Natural Language Processing which are discussed below.

From the point of view of it's users, Partridge will assist researchers in two
ways. The system will provide filtering of papers based upon their
specific domain (i.e. is the paper primarily concerned with methodology within
an experiment in chemistry or is it about Ethics in Psychological studies?) and
their result, whether the paper yielded positive, negative or inconclusive
evidence for a hypothesis. Depending upon the time constraints of the
project, it is hoped that Partridge will also offer a user 'profiling' system
that provides recommendations for researchers based on their reading history.
This may help Partridge users find relevant papers more quickly or find
research that they may have otherwise overlooked.

Despite the existence of several complex information retrieval systems
already in this domain, it is hoped that Partridge will offer a service that
proves to be more effective and useful for locating relevant scientific papers
than any counterparts.

\subsection{Prior Art}


\section{Current Progress}

\section{Planning}

\pagebreak
\bibliographystyle{IEEEannot}
\bibliography{report}

\end{document}
