% Outline Project Specification Report

%The document is a report
\documentclass[12pt,a4paper]{article}

%define horizontal rule
\newcommand{\HRule}{\rule{\linewidth}{0.5mm}}

\usepackage{fullpage}

%use the listings package
\usepackage{listings}
%use the English language
\usepackage[english]{babel}
%use graphics
\usepackage{graphicx}
%use wrap figures
\usepackage{wrapfig}
%geometry stuffs
\usepackage{lscape}
%use natbib bibliography package
%\usepackage[numbers]{natbib}
%use harvard bibliography package
%\usepackage{harvard}	
%use captions
\usepackage{caption}
%use multi-row tables
\usepackage{multirow}
\usepackage{url}
\usepackage{subcaption}

\begin{document}

%use harvard citations
%\citationstyle{agsm}

%include the title page
\newcommand{\Revision}{76c5729}

\begin{titlepage}
 
\begin{center}
 
 
% Upper part of the page
\includegraphics[width=0.20\textwidth]{./Gerald_G_Man_in_Suit.png}\\[1cm]
 
\textsc{\LARGE Aberystwyth University}\\[1.5cm]
 
\textsc{\LARGE Industrial Year Report}\\[0.5cm]
 
 
% Title
\HRule \\[0.4cm]
{ \huge \bfseries Blue Harvest - My Year at IBM }\\[0.4cm]
 
\HRule \\[1.5cm]

 % Author and student ID
\begin{minipage}{0.4\textwidth}
\begin{flushleft} \large
\emph{Author:}\\
James \textsc{Ravenscroft}
\end{flushleft}
\end{minipage}
\begin{minipage}{0.4\textwidth}
\begin{flushright} \large
\emph{Student ID:} \\
090407039
\end{flushright}
\end{minipage}

\vfill
 
% Bottom of the page
{\large \today}
 
\end{center}
 
\end{titlepage}



%some definitions for paragraph layout stuff
\setlength{\parindent}{0pt}
\setlength{\parskip}{1.5ex plus 0.5ex minus 0.2ex}

\tableofcontents

\pagebreak

\section{Project Summary}

Partridge is a web-based tool designed to assist in information processing and knowledge
acquisition within the domain of scientific research.

Since the advent of the 'Digital Age' and the ability of computers to copy and
reproduce information for a negligible cost, the amount of information
available to researchers has been increasing drastically.  A 2009 study by B-C
Bj\"{o}rk suggests that approximately 1.4 Million papers were published in the
year 2006 alone\cite{bjork2009}. Moreover, the growing popularity of Open Access
publishing (making papers available for free online\cite{Suber2012}) across
most scientific disciplines\cite{bjork2009}\cite{harnad2004comparing} is
providing researchers with an even larger volume of information to be
processed. As available information increases, relevant material becomes
progressively more difficult to find and the need for an automated information
retrieval tool more apparent.

Partridge aims to autonomously process as many scientific papers as possible to
facilitate researchers who would otherwise be required to manually read each
paper themselves. This should reduce the amount of information that the reader
is required to process themselves, thereby speeding up the research process.
Partridge will achieve this through the use of several existing techniques in
the field of Natural Language Processing which are discussed below.

From the point of view of it's users, Partridge will assist researchers in two
ways. The system will provide filtering of papers based upon their
specific domain (i.e. is the paper primarily concerned with methodology within
an experiment in chemistry or is it about Ethics in Psychological studies?) and
their result, whether the paper yielded positive, negative or inconclusive
evidence for a hypothesis. Depending upon the time constraints of the
project, it is hoped that Partridge will also offer a user 'profiling' system
that provides recommendations for researchers based on their reading history.
This feature should help users find relevant papers more quickly or find
research that they may have otherwise overlooked.

There are already several tools that help researchers manage the vast library
of internet journals available on the internet. Search engines such as Google
({\url{http://www.google.com/}}), and social citation management tools such as
CiteULike ({\url{http://www.citeulike.org}}), do offer some assistance in
tracking down relevant information. However, these tools are often too general
or rely upon the user knowing exactly what keywords to search for before
carrying out the search. These drawbacks are further discussed in Section
\ref{sec:prior_art} below.

To overcome the drawbacks of these existing systems, Partridge will make use of
several modern Natural Language Processing (NLP) techniques.  NLP enables the
automated extraction of meaningful information from texts written in human
languages such as English or French. There have already been several papers on
using NLP for detecting emotions in suicide notes\cite{citeulike:11077287}, the
genre of a web page on the World Wide Web\cite{citeulike:11288938} and
emotional polarity of a phrase or sentence\cite{Wilson05Polarity}. Liakata et
al. have also used NLP techniques to classify sentences within scientific
papers to determine what scientific concept they relate to (i.e. does this
sentence cover background information or is it a part of the
hypothesis?)\cite{citeulike:10444769}. Partridge will build upon and make use
of these existing applications of NLP to filter and retrieve data from
scientific papers in a novel way.

\section{Current Progress}

The Partridge project has been underway since the beginning of October. The
following section looks at related works and how Partridge compares to them,
some prior investigation and prototyping work that has already been carried
out.


\subsection{Related Works}
\label{sec:prior_art}

There are already many existing systems for finding and filtering information
on the World Wide Web. Search engines are very useful for information retrieval
in this very large and generalised search domain. Most people have heard of Google (\url{http://wwww.google.com}),
Yahoo (\url{http://www.yahoo.co.uk}), Bing (\url{http://www.bing.com}) and Ask
(\url{http://www.ask.com}). There are many more similar systems available for free
general use across the internet. They all present very similar user interfaces
(as shown in Figure \ref{fig:search_interfaces})
in which users are asked to supply keywords that might be linked to relevant documents 
and the search engine returns a list of Uniform Resource Locators (URLs) that they 
consider to match the user's query. 

\begin{figure}[!ht]
        \centering
        \begin{subfigure}[b]{0.50\textwidth}
                \centering
                \includegraphics[width=\textwidth]{images/ask_front.png}
                \caption{Ask.com}
                \label{fig:ask_interface}
        \end{subfigure}%
        \begin{subfigure}[b]{0.50\textwidth}
                \centering
                \includegraphics[width=\textwidth]{images/bing_front.png}
                \caption{Bing.com}
                \label{fig:bing_interface}
        \end{subfigure}\\
        \begin{subfigure}[b]{0.50\textwidth}
                \centering
                \includegraphics[width=\textwidth]{images/google_front.png}
                \caption{Google.com}
                \label{fig:google_interface}
        \end{subfigure}%
        \begin{subfigure}[b]{0.50\textwidth}
                \centering
                \includegraphics[width=\textwidth]{images/yahoo_front.png}
                \caption{Yahoo.com}
                \label{fig:yahoo_interface}
        \end{subfigure}%
        \caption{4 popular search engine interfaces}\label{fig:animals}
        \label{fig:search_interfaces}
\end{figure}


Search engines are helpful in locating pages and websites within the World Wide
Web. Unfortunately, the problem space they deal with is usually too big for
them to find scientific papers and journals given a set of keywords. Internet
search engines index a huge proportion of irrelevant information compared to
useful information\cite{Berghel1997}, and as a result, even relatively specific
queries such as effects of gravity on rockets" yield millions of results (as
shown in Figure \ref{fig:rocket_results}). 

\begin{figure}[!ht]
\includegraphics[width=0.80\textwidth]{images/space_rocket_query.png}
\caption{{Google showing over 7M results for ``effects of gravity on rockets"}}
\label{fig:rocket_results}
\end{figure}

Partridge offers an advantage over these mechanisms as it will specifically
index research papers rather than attempting to index the whole Internet.
This means that there should be a higher proportion of useful information as
output compared to the output of an Internet Search Engine.

There are also a number of search and indexing systems that specifically look
for scientific papers as opposed to web pages. One of the most publicised and
well known paper search system is Google Scholar
(\url{http://scholar.google.com}).  As can be seen in Figure
\ref{fig:scholar_basic}, This is an adaptation of Google's general search
engine (discussed above) to specifically index and search scientific papers.
Google also offers advanced query options specific to Scholar that allow
searching by author, year and for words that occur only in the document title
as shown in figure \ref{fig:scholar_advanced}. Whilst this does deal with the
problem of `information overload' and provides even more fine control over the
information returned from searches,  the user is still required to have a very
good idea of what they are looking for in terms of keywords and/or specific
authors. It is possible that a user would not know what they are looking for
until they've seen it. 

\begin{figure}[!hbt]
        \centering
        \begin{subfigure}[b]{0.50\textwidth}
                \centering
                \includegraphics[width=\textwidth]{images/googlescholar_front.png}
                \caption{Google Scholar's General front page}
                \label{fig:scholar_basic}
        \end{subfigure}%
        \begin{subfigure}[b]{0.50\textwidth}
                \centering
                \includegraphics[width=\textwidth]{images/googlescholar_advanced.png}
                \caption{Advanced search features}
                \label{fig:scholar_advanced}
        \end{subfigure}

        \caption{Google Scholar's user interface}
        \label{fig:scholar_interface}
\end{figure}

Partridge will provide the option to filter papers by subject and it is hoped
that the system will also provide user-specific recommendations by profiling
them through their reading history. This will make it easier for users to find
relevant papers without knowing exactly what keywords they want to search for.



\section{Planning}

\pagebreak
\bibliographystyle{IEEEannot}
\bibliography{report}

\end{document}
