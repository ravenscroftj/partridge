% Outline Project Specification Report

%The document is a report
\documentclass[12pt,a4paper]{article}

%define horizontal rule
\newcommand{\HRule}{\rule{\linewidth}{0.5mm}}

\usepackage{fullpage}

%use the listings package
\usepackage{listings}
%use the English language
\usepackage[english]{babel}
%use graphics
\usepackage{graphicx}
%use wrap figures
\usepackage{wrapfig}
%geometry stuffs
\usepackage{lscape}
%use natbib bibliography package
%\usepackage[numbers]{natbib}
%use harvard bibliography package
%\usepackage{harvard}	
%use captions
\usepackage{caption}
%use multi-row tables
\usepackage{multirow}
\usepackage{url}

\begin{document}

%use harvard citations
%\citationstyle{agsm}

%include the title page
\begin{titlepage}
 
\begin{center}
 
 
% Upper part of the page
\includegraphics[width=0.20\textwidth]{./Gerald_G_Man_in_Suit.png}\\[1cm]
 
\textsc{\LARGE Aberystwyth University}\\[1.5cm]
 
\textsc{\LARGE Industrial Year Report}\\[0.5cm]
 
 
% Title
\HRule \\[0.4cm]
{ \huge \bfseries Blue Harvest - My Year at IBM }\\[0.4cm]
 
\HRule \\[1.5cm]

 % Author and student ID
\begin{minipage}{0.4\textwidth}
\begin{flushleft} \large
\emph{Author:}\\
James \textsc{Ravenscroft}
\end{flushleft}
\end{minipage}
\begin{minipage}{0.4\textwidth}
\begin{flushright} \large
\emph{Student ID:} \\
090407039
\end{flushright}
\end{minipage}

\vfill
 
% Bottom of the page
{\large \today}
 
\end{center}
 
\end{titlepage}



%some definitions for paragraph layout stuff
\setlength{\parindent}{0pt}
\setlength{\parskip}{1.5ex plus 0.5ex minus 0.2ex}

%\tableofcontents

\pagebreak

\section{Project Description}

In academic research, there are a huge abundance of scientific papers available for
consumption for any given scientific pursuit. In 2010, the famous journal,
PLOS One (\url{http://plosone.org}), reached the 10,000 papers published milestone \cite{PlosOne2010}. If you 
are doing background reading for a paper you are authoring, or looking for 
research on a specific topic that interests you, the vast quantity of data available can often be
quite daunting and choosing which papers to read very difficult. This problem
is further exacerbated when there is a time limit on the research to be carried
out. Partridge aims to aid the reader in finding documents relative to their
interests through the use of Artificial Intelligence techniques.



\section{Work to be tackled}

The main body of work to be tackled during the project is the Natural Language
Parsing (NLP) element of the application. 



\section{Project Deliverables}

\pagebreak


\bibliographystyle{IEEEannot}
\nocite{*}
\bibliography{report}

\end{document}
