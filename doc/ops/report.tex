% Outline Project Specification Report

%The document is a report
\documentclass[12pt,a4paper]{article}

%define horizontal rule
\newcommand{\HRule}{\rule{\linewidth}{0.5mm}}

\usepackage{fullpage}

%use the listings package
\usepackage{listings}
%use the English language
\usepackage[english]{babel}
%use graphics
\usepackage{graphicx}
%use wrap figures
\usepackage{wrapfig}
%geometry stuffs
\usepackage{lscape}
%use natbib bibliography package
%\usepackage[numbers]{natbib}
%use harvard bibliography package
%\usepackage{harvard}	
%use captions
\usepackage{caption}
%use multi-row tables
\usepackage{multirow}
\usepackage{url}

\begin{document}

%use harvard citations
%\citationstyle{agsm}

%include the title page
\begin{titlepage}
 
\begin{center}
 
 
% Upper part of the page
\includegraphics[width=0.20\textwidth]{./Gerald_G_Man_in_Suit.png}\\[1cm]
 
\textsc{\LARGE Aberystwyth University}\\[1.5cm]
 
\textsc{\LARGE Industrial Year Report}\\[0.5cm]
 
 
% Title
\HRule \\[0.4cm]
{ \huge \bfseries Blue Harvest - My Year at IBM }\\[0.4cm]
 
\HRule \\[1.5cm]

 % Author and student ID
\begin{minipage}{0.4\textwidth}
\begin{flushleft} \large
\emph{Author:}\\
James \textsc{Ravenscroft}
\end{flushleft}
\end{minipage}
\begin{minipage}{0.4\textwidth}
\begin{flushright} \large
\emph{Student ID:} \\
090407039
\end{flushright}
\end{minipage}

\vfill
 
% Bottom of the page
{\large \today}
 
\end{center}
 
\end{titlepage}



%some definitions for paragraph layout stuff
\setlength{\parindent}{0pt}
\setlength{\parskip}{1.5ex plus 0.5ex minus 0.2ex}

%\tableofcontents

\pagebreak

\section{Project Description}

For scholars carrying out a literature review or researching a specific topic
the vast quantity of research papers available on the internet can often be quite daunting;  
with the advent of the `Information Age', choosing which papers to read 
has become very difficult. This problem is exacerbated when there is a 
time limit to the research. Partridge aims to aid the 
reader in finding documents aposite to their interests through the use of 
Artificial Intelligence techniques. As well as making research processes 
more efficient and saving time, the program may also help the scholar to find
papers that they may otherwise not have read.

Partridge will use a web interface to allow users to specify particular
research topics. This can be done through manual entry of interests and types of 
report or by allowing the application to collate the papers that the user 
has read and to learn what is relevant to them. By appropriating this information,
Partridge will use learning algorithms to examine and classify the available
corpus of literature and make recommendations as to which articles the
scholar would find most helpful. 

\section{Work to be `tackled':}
Despite the need for a web interface, the main body of the work in this project 
comes from the backend of the system and the Artificial Intelligence that drives it.

Initial research must be undertaken  into Natural Language
Parsing techniques and how different features of a text can be used to
classify papers. Furthermore, clear identification of the attributes which
allow the system to understand the differences between texts is vital.

Once these features are understood, a learning algorithm will be developed to
analyse scientific papers and to store information about the differences between 
them. The system will make use of some third party software (SAPIENTA) to help
with this process. Additionally, knowledge from CS36110 Machine will be used in
this project.

The first iteration of the system should the search for  that match specific criteria,
corresponding directly to `features' identified using the aforementioned AI
algorithms. Further iterations could be used  to build a profile based upon the user's 
reading history and used to make suggestions for further.

\section{`Project Deliverables'}

Although a formal 'XP' methodology will not be adopted since this is an
individual project, the project will be developed using an agile approach. It will be built in an
iterative manner. An initial release of the product will be made as soon as
possible. Subsequent releases will be made in iterations of two months.

Deliverables include:
\begin{itemize}
    \item Initial version of the system with restricted functionality; This
    would  use simple search and filtering techniques to find relevant research
    papers;
    \item Progress report evaluating the initial product and outlining ongoing
    development of the Artificial Intellegince underlying the results
    classifier module;
    \item Release two of the product, including result classifying behaviour;
    \item Demonstration of the artificial intelligence in the product;
    \item Following iteration three, the next release of the product will show
    evolved classification behaviour.
    \item Final report and release of software.
\end{itemize}


\renewcommand{\refname}{Initial Bibliography} 
\bibliographystyle{plain}
\nocite{*}
\bibliography{report}

\end{document}
