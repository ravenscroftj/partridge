% IY Report

%
% WORD COUNT COMMAND
% 
%  untex -m -e -o report.tex | wc -w
%

%The document is a report
\documentclass[12pt,a4paper]{article}

%define horizontal rule
\newcommand{\HRule}{\rule{\linewidth}{0.5mm}}

\usepackage{fullpage}

%use the listings package
\usepackage{listings}
%use the English language
\usepackage[english]{babel}
%use graphics
\usepackage{graphicx}
%use wrap figures
\usepackage{wrapfig}
%geometry stuffs
\usepackage{lscape}
%use natbib bibliography package
\usepackage{natbib}
%use harvard bibliography package
%\usepackage{harvard}	
%use captions
\usepackage{caption}
%use multi-row tables
\usepackage{multirow}
\usepackage{url}

\begin{document}

%use harvard citations
%\citationstyle{agsm}

%include the title page
\begin{titlepage}
 
\begin{center}
 
 
% Upper part of the page
\includegraphics[width=0.20\textwidth]{./Gerald_G_Man_in_Suit.png}\\[1cm]
 
\textsc{\LARGE Aberystwyth University}\\[1.5cm]
 
\textsc{\LARGE Industrial Year Report}\\[0.5cm]
 
 
% Title
\HRule \\[0.4cm]
{ \huge \bfseries Blue Harvest - My Year at IBM }\\[0.4cm]
 
\HRule \\[1.5cm]

 % Author and student ID
\begin{minipage}{0.4\textwidth}
\begin{flushleft} \large
\emph{Author:}\\
James \textsc{Ravenscroft}
\end{flushleft}
\end{minipage}
\begin{minipage}{0.4\textwidth}
\begin{flushright} \large
\emph{Student ID:} \\
090407039
\end{flushright}
\end{minipage}

\vfill
 
% Bottom of the page
{\large \today}
 
\end{center}
 
\end{titlepage}



%some definitions for paragraph layout stuff
\setlength{\parindent}{0pt}
\setlength{\parskip}{1.5ex plus 0.5ex minus 0.2ex}

\tableofcontents

\pagebreak

\section{Introduction to IBM and its Organisational Structure}

\subsection{100 Years of Innovation}
2011 marked the centennial anniversary for IBM as a business.  The company was
founded in 1911 as the result of a merger between three $19^{th}$-Century
companies: the Tabulating Machine Company, the International Time Recording
Company and the Computing Scale Company of America. In 1914, Thomas J. Watson
Sr. joined the company shifting the direction of the company to multinational
technology consultation.

\subsection{IBM Today}

\begin{figure}[!ht]
\includegraphics[width=\textwidth]{./company.png}
\caption{A diagram of IBM's internal corporate structure}
\end{figure}
Today, IBM remains one of the largest technology corporations on the
international market, employing over 400,000 people across more than 170
countries. The current CEO, Ginni Rometty, replaced Sam Palmisano, former CEO of
nine years, in January 2012. Rometty has been with IBM since 1981, joining the
firm as a systems engineer. She held the role of senior vice president and group
executive for sales, marketing and strategy for three years previous to her
promotion to CEO. Rometty was also the first female CEO of the firm.

IBM is mainly based in the United States and remains a US-registered company.
However, the company operates on a global scale with three separate business
units servicing the Americas, Europe with the Middle East and Asia (EMEA),   
Asia and the Pacific. Each regional unit is divided up into operational units
based on function: Global Business Services (GBS) provide consulting and
procurement services; the Software Group (SWG) maintains and services IBM’s
software portfolio. There are also separate divisions for research, sales and
distribution and for corporate strategy and management.

SWG is divided up into teams, each one servicing one of IBM’s software products
from the portfolio. Each of these teams can be further subdivided into teams of
developers, testers and customer service engineers. During my placement, I
worked as part of the CICS Testing infrastructure team on the IBM Hursley campus
(for further details, see Section \ref{sec:myteam}).

\subsection{Hursley}
Hursley House is a mansion constructed in the $18^{th}$ Century in rural
Hampshire by William Heathcode, during the reign of King George I. The house was
used in World Wars I and II as a military hospital and as a base of operations
for Vickers Aviation who produced the world famous Spitfire, during their time
at at the site.

In 1958, IBM purchased the site and started using Hursley House for research and
development purposes and as a temporary home. In 1963, the company purchased the
100 acres of land surrounding the house and began to establish a more permanent
base, erecting a modern office complex in the house gardens. The main house no
longer contains any development teams, remaining home only to the senior staff
members and Hursley Lab Director, John McLean.

The rest of the site is made up of office blocks connected internally by glass
walkways. Each block is mainly devoted to a specific software product or a
family of products. Hursley currently houses development and service teams for
many of IBM’s middleware software packages including WebSphere products such as
the MQSeries message propagation server and the WebSphere Application Server
(WAS), CICS and CICS Explorer, and IBM Java Runtime.

Hursley also houses Emerging Technology Services where experimental technology
is tested by IBM and used to develop prototype solutions for clients. ETS is the
only Research and Development department on site. All of the other departments
at Hursley are specifically for development and service of existing software and
fall under the management of IBM’s Software Group business unit.

\section{IBM's Technical and Application Environment}

\subsection{Office and Development Environment}
Upon starting my job at IBM I was given an IBM Thinkpad T60p for my every-day
work and business. Thinkpad laptops and docks are standard issue within the
company as they allow employees to work using external peripherals such as
full mouse and keyboard and external monitor in their office, to take their
laptops with them to meetings and work from home out of hours.

All interns were given identical laptop hardware which came with a pre-installed
Windows XP image customised with IBM utilities . Locks and bags were also
distributed to all interns and we were encouraged to lock our laptops to our
desks at all times, since many of us were based in open-plan offices. 

As part of the standard IBM desktop environment, all employees were given a copy
of Lotus Notes for internal communication and organisation. Notes was also used
for organisation of meetings and propagation of meeting invitations across the
site. All employees are also given a copy of Lotus Sametime, an instant
messaging client for communication within large businesses. This made contacting
colleagues for help and assistance a lot easier since you could instantly talk
to anyone currently logged on to a computer terminal connected to the IBM
network.

For editing documents and presentations, all interns were given a copy of IBM 
Lotus Symphony. Symphony is a fork from the OpenOffice.org office suite, an open
source office suite originally started by Sun Microsystems before being bought
out by Oracle. I found Symphony to be quite unstable and was advised to use 
LibreOffice instead, since the document file formats are compatible. In January,
it was announced that IBM will stop developing Symphony, instead contributing to
the OpenOffice project \cite{SunsetSymphony}.

For programming and version control purposes, IBM supplied me with a copy of
Rational Team Concert for System Z. This is a suite of software that sits on top
of the Open Source Eclipse IDE providing version control capability and a
graphical interface for editing data remotely on IBM mainframe systems running
zOS. The package also provides extensions to eclipse for the syntax highlighting
and auto-completion for the programming languages REXX and PL/1.

To complement the Eclipse development suite, my team also use a set of
commandline utilities for development and administration of the test machines.
The x3270 \cite{x3270Site} terminal emulator was used for remote administration
of the team's z/OS test machines whilst SSH and FTP could be used for remote
interaction with the mainframe's Unix System Services interface.

Despite being given a Windows XP installation to start with, I was advised by my
team to install IBM's Ubuntu-based Linux distribution, OCDC, instead. Most of my
team used Linux themselves and I found that many of IBM's software packages ran
a lot more smoothly on a Linux installation than in Windows.

\subsection{Development Server Environment}
%---------------------------------
%YOU NEED A CITATION HERE.
%
%-------------------------------
Much of IBM's software is still designed for z/OS-based mainframe machines, the 
current generation of which, are some of the most powerful enterprise hardware
solutions available in the global marketplace. CICS, the product that I helped
to develop, is one such software package.

\subsubsection{CICS}
During my placement, I was tasked with working on IBM's CICS Transaction Server.
CICS is a commercial middleware package that runs on mainframes and provides
secure and stable transaction handling. CICS was originally released in 1969 and
developed and maintained by a team at the IBM Development Centre in Des Plaines,
Illinois. In earlier versions, CICS was bundled for free with IBM mainframe
hardware at no extra cost. IBM customers were quick to understand the value of
CICS which quickly became one of IBM's most important software packages, helping
to raise over \$60 Billion in hardware revenue. However, IBM Executives did not
see the value of the software package and CICS was sent to Hursley to be
developed and maintained to allow other software packages to be developed in the
US. Despite this, continued development at Hursley park has enabled CICS to
continue to be one of IBM's flagship software packages.  In 2004, IBM reported
that CICS is now used by over ninety per cent of the Fortune 500 companies and many
Government entities now also use the product.

CICS is provided as a Software Development Middleware and users write their own
applications to make use of its features and API. IBM provide bindings and
compilers to enable applications to interact with CICS through languages such as
C, COBOL and PL/1. More recently, Java support has been integrated into CICS,
allowing the use of OSGi bundles within the application environment using IBM's
JRE.

\subsubsection{z/OS}
The IBM Mainframes mainly run the proprietary operating system z/OS. z/OS is the
spiritual successor to IBM's OS/390 and provides many of the same features 
and functions as its predecessor. The operating system is designed to provide
enterprise security and performance on mainframe systems.

For developers only familiar with UNIX and Windows-style operating systems, z/OS 
requires a very steep learning curve. The operating system is unique in its
approach to storage, memory management and security. Programs cannot be 
directly executed on z/OS and instead must be bootstrapped through the use
of Job Control Language (JCL). 

\subsubsection{DB2 and MQ}
IBM's policy of reusing much of their own software within new products under
development meant that despite my position in the CICS team, I was also exposed
to many other products in their software portfolio, particularly during my work
on JAT+ (see Section \ref{sec:jatp}).

\section{Job Role}
\label{sec:myteam}
My job role was to serve as part of the CICS Test Infrastructure Team who wrote
and maintained software test infrastructure systems and tools. Our team worked
alongside the CICS Functional Verification (FV) team who were responsible for
testing new CICS code to make sure that it worked as documented before it was
released to the public. 

\subsection{RTS}
The FV team used an internal product called Regression Testing Suite (RTS) to
run automated suites of test scripts against the product and check for
regressions as well as any new defects in the code. The infrastructure team and
I were responsible for servicing the mainframes that RTS ran on as well as
writing new features and fixing defects for the RTS framework itself.

Upon my initial placement within the team, my main job role was to help service
and support RTS and related test systems. My day to day role involved
programming in Perl and the IBM scripting language REXX. REXX is an interpreted
language which is not widely used in the larger computing world. However, REXX
is a very popular language in mainframe environments and is the primary
development language for RTS. 

RTS can be accessed through a web interface or from a mainframe terminal.
Testers provide a script which specifies the nature of the CICS instances to be
set up, a set of operations to be carried out by the software, and then any
steps required at shutdown. Scripts can be triggered through either of the above
interfaces or scheduled to run on an automated basis.

My duties were often to resolve defects raised against specific parts of the RTS
code-base and occasionally to implement new functionality as required by testing
team members. Unstable code had to be thoroughly tested on the RTS development
servers before being deployed to the primary FV test servers to prevent CICS
tests from failing due to faulty RTS code. The team was also prohibited from
making any changes to RTS on a Friday afternoon to prevent automated test runs
scheduled over the weekend from failing due to unstable infrastructure code.
Stable RTS code that functioned correctly under test for a week was placed in
version control.

The team used a legacy IBM version control system for RTS code. Although the
preferred version control system for IBM systems is Rational Team Concert, 
RTS shared several components with one of the MQ testing utilities. The code
was eventually moved over to RTC towards the end of my placement when the MQ
team also migrated to the new version control tool.

\subsection{COMP4}
After a few months, I was assigned to a new Java project under joint management
of the CICS Infrastructure and the Functional Verification teams. The project was
started by a fellow intern and provided a graphical front end for the compilation
of CICS supported languages on the testing mainframes. 

Our system was named COMP4 as it replaced an existing interface named COMP3. It was
designed as an Eclipse plugin. It added a context menu option to "Compile with
COMP4" when inspecting any CICS compatible source file. The system then filled
out a JCL template for execution on the mainframe and sent the job to the 
remote system to be run. Once a specified period of time had passed, the job 
result was collected from the mainframe and displayed to the user in the UI.
The system was also designed in a multi-threaded way which meant that concurrent
compilation of several source files was supported natively by our plugin.

Not only was COMP4 a real test of my Java and OSGi programming knowledge, but it
provided me with an opportunity to hone my management and communication skills.
I was brought in to supervise Daniel, my colleague on the project, and help to
organise the architecture of the application. Daniel was not confident at
programming or application design. Although it was quite tempting to simply do
the programming myself and complete the job on my own, I wanted to facilitate 
Daniel in his learning. I was able to tutor him and I
helped him to improve his programming skills. This in turn, allowed me to put into
practice my mentoring skills which I'd previously acquired from my work as a
demonstrator at Aberystwyth University.

Daniel and I were able to make several critical changes to the way that COMP4
had been designed to speed up development. Our managers had suggested that we
use ANTz, an IBM extension to the Java ANT build engine, for the COMP4 backend.
ANTz is very complicated and would have required a steep learning curve for both
of us. The product was also very heavyweight and would have been over-complicated
for COMP4's needs. Therefore, Daniel and I decided to engineer our own back end
using Apache Velocity as a templating engine and submitting pre-generated JCL
jobs to the mainframe. This solution proved to be much more apt than ANTz and we
received much acclaim for our work. The change also allowed us to release our
tooling a month ahead of the specified deadline, again helping both of our
reputations within the department.

\subsection{JAT+}
\label{sec:jatp}
When I arrived at IBM in July, RTS was already an aging system and had been used
for testing CICS and Functional Verification for several years. RTS was also
very unstable and the complicated nature of the shared source repository meant
that there were often many problems with the system. After my success on the
COMP4 project, I was asked to help design and implement a replacement test
system for CICS that would be more stable, efficient and easier to use. 

It was decided that the project should be written in Java and would make use of
existing technology such as Eclipse and J-Unit, the widely used Java unit
testing framework. Like COMP4, the product was designed as an Eclipse plugin
since most of the CICS developers use Eclipse for their day to day work and are
familiar with the IDE's interface. It was also designed to interface with
Eclipse's existing JUnit interface since many of the CICS developers use JUnit
already and are familiar with the framework. Reuse of JUnit also allowed us to
speed up development and avoid rewriting complicated test framework
functionality. It was decided that our product would be called Java Test Plus
(JAT+).

JAT+ facilitates the execution of CICS tests on a local instance of Eclipse by
running sets of remote commands over the network. Tests are provided in a JUnit
test class style with some special annotations and fields that specify which
CICS instances on the network to test etc. The plugin can automatically
provision a testing environment from a pool of available server resources.
Alternatively, the tester can manually set up their testing environment on the
mainframe to save time or test custom CICS functionality.

JAT+ also runs remotely on an automation server and CICS tests can be sent to
the server in OSGi bundles and tested as part of a regular schedule. The results
of remote tests can still be viewed in the eclipse JUnit results view as if they
had been locally executed.

My role on the project varied throughout my placement. I started off as a lead
developer on the back end of the project, building classes that facilitated the
initial setup and provisioning stages of test execution. This required me to
work closely with mainframes and networking libraries in order to execute the
necessary remote jobs and build CICS regions. Luckily, my work with COMP4
prepared me for this task. Once the initial framework was in place, I was asked
to develop several Eclipse User Interface components for the project to make it
more usable by testers. I implemented several new views and context menu options
to make using the plugin more convenient and usable.

Towards the end of my placement, it was decided that JAT+ would be adopted as 
the primary CICS testing framework. Therefore, all of the existing automated
tests for CICS would need to be translated from the RTS format to the JAT+ 
format. There was a very large quantity of automated test suites that had 
already been written for RTS; it was decided that the conversion process should
be automated to minimise the amount of time spent converting the tests. My last
role before leaving IBM in August was to design and implement the conversion 
engine that would translate each of the tests. 

\subsection{Extra Roles \& Giveback}
As well as their main job role, IBM employees are expected to contribute up to 
50\% of their working day to side-projects. This practice is known internally
as "Giveback." During my placement I was involved in several Giveback projects
from the very beginning of my placement.

\subsubsection{PHP \& Java classes}
Towards the beginning of my placement, a notice was sent out requesting
volunteers to teach other IBM employees on a range of technical topics including
basic introductions to Java and PHP. I had previous experience of teaching Java
from my work as a Demonstrator at Aberystwyth University, and I had a background
in writing PHP applications from when I first started building websites as a
teenager. I volunteered and was accepted as a teacher/lecturer on both topics.

The PHP course I taught ran over a period of 4 weeks. I prepared and delivered
each presentation without any guidance aside from a briefing on the target
audience and their skill level. The first lecture was a brief introduction to
PHP and its main applications. The second lecture was a brief summary on how to
set up PHP in an Apache environment on your computer and to create a hello world
web page. The final two sessions summarised some standard PHP programming
practices and I provided worksheets and solutions to the audience to help them
to get to grips with the language more effectively. There were also Q\&A
sessions at the end of each lecture and I was able to help many of the audience
members in person.

The Java course ran in a slightly different way. Since more people seemed to be
familiar with the language, there were as many volunteer tutors as there were 
students. It was decided that the group would split into two: Those with no 
technical background at all who wanted to learn to program, and those with 
a technical background in other languages who wanted to learn the language Java.
I was asked to help teach the more technical group. Weekly sessions were run for
approximately 12 weeks at the start of our placement. We also set a programming
challenge for the students to help them to get used to the language and to 
prompt them to ask any questions about topics that confused them.

To begin with, each tutor took it in turns to present a Java concept to the
group. All tutors attended the sessions and were able to add their comments to
the material if they felt that the presenter had forgotten anything or hadn't
explained something in detail. After a period of several weeks, the students 
got used to the core language concepts and the group evolved more into a 
reading/common interest group. We would meet up weekly for a Q\&A session and 
to offer each other tips and advice for Java programming in our day jobs. 

After a few weeks of Q\&A, the students decided that they had learned enough 
about Java to be confident using it in their day jobs and the sessions came 
to an end.

\subsubsection{Activity Centre Day}
As well as technical activities, IBM Giveback also offers employees the 
opportunity to go out into the community and do something completely different.
In August 2011, I was asked to go with a team of approximately 50 other IBM
employees to visit a charity-run activity centre in the New Forest, Hampshire.
We spent the day carrying out maintenance tasks at the center including 
cutting grass, trimming back bushes and weeding. Once we'd finished our duties
we were allowed to do some of the activities including rock climbing and raft
building.

\subsubsection{Web Design Skills Workshop}
In March, I was contacted by one of my colleagues to ask for my assistance in
the organisation and delivery of a web design workshop for a small group of
secondary school children because of my knowledge of the PHP language and HTML. 
Three interns, including myself, went out to Romsey School, near Hursley,for 
half a day to deliver the workshop to the children and provide an hour of 1:1
tuition. This experience allowed me to build upon my presentation and teaching
skills which I had already developed from my PHP Lectures and to learn on how
to present to a younger audience more effectively.

\subsubsection{Map Tool Project}
IBM Hursley is a large site with many offices and an inconsistent coordinate
system for navigating around the campus. There is currently a web tool which
provides a map of the site and can be queried to pinpoint the location of a
specific person or room at Hursley. Unfortunately, this tool is not very 
intuitive to use and the map output is difficult to understand. Some colleagues
and I decided that there was a room for improvement and set up a Giveback 
scheme in order to rewrite and improve the Hursley Map tool.

The team, lead by myself, consisted of 7 interns at the Hursley site. We met
weekly and spent half a day discussing and working on the new tool. The 
project was a great opportunity for me to meet many IBM employees as full 
analysis of the project required us to consult the on-site architect, the
previous owner of the map tool and the current web admin for the Hursley 
intranet. 

We were told that site plans were available in a common format for all IBM
offices in the UK and that the new map tool could potentially be used by IBM on
a national or even global scale if it was successful. It was decided that the
new project would be written in PHP since most of the team were familiar with
the language and that we would use some mapping frameworks that are already
available to accelerate development.

Unfortunately, the 12 month time limit on our placements at IBM meant that we
were unable to complete the map tool before we left the company. Instead, we
found a group of interns from the 2012 intake to take over the project and I
ran a couple of short handover sessions to help familiarise them with the 
project codebase.

\section{Critical Evaluation}

During my year at IBM I learnt a lot about programming, working in large 
development communities and myself. Although I found the work challenging
to begin with, I quickly adapted to my new environment and was able to  
become a valued member of my team very quickly.

My first few weeks at work were incredibly difficult; the initial learning
curve at the time of joining my team was near vertical. My first few tasks
after joining the CICS Infrastructure Team were to make minor changes to 
some of the RTS software components. Having never seen a mainframe terminal
or worked with CICS, REXX, MVS or z/OS before, I found this task brutally 
challenging. This feeling was exaggerated by reports from some of my friends
that they were working on Java projects on Linux and Windows - environments
that I was a lot more familiar with at the beginning of my placement. I was
informed at the beginning of my placement that if I was not happy with my 
job, I could be found a new role elsewhere in the company. Although I was
tempted at this stage, I persevered and rose to the challenges that I faced.
After a few weeks of struggling and counsel from my team, I started to 
become proficient at my role and enjoyed my job a lot more. The knowledge I
acquired during my first few weeks at IBM  has definitely given me a much
stronger background in IBM software and hardware and has provided me with 
some hugely desirable skills for any employers that use IBM products. After
my experience I think that IBM should certainly consider more avenues of 
support and initial new-starter education so that joining a new team is not
so much of a challenge and does not discourage new employees too much.

IBM's software product portfolio now consists largely of Eclipse-based 
graphical applications are written in Java. Although Java is a powerful
and flexible language, it was surprising to see how much faith the company
are putting into the language. Even mission critical systems like CICS 
that should be programmed in the lowest level languages possible to 
maximise speed and efficiency are being infiltrated by VM-based memory-
hungry Java \cite{cicsJava}. Personally, I feel that this is a
mistake and that development of such applications should continue in 
low level or Ahead-of-time compiled languages like C and COBOL. Unfortunately,
as long as universities continue to train students to be Java programmers, it
would be difficult to obtain a set of graduate employees with the necessary 
skills to maintain a COBOL or even a C-based system. However, as companies like
IBM move more towards apprenticeships, they are able to train new recruits as
they wish.

IBM have a dual management system for interns and graduates who join the 
company. I was assigned a line manager who oversaw my day to day work and
dealt with any problems I might have in my day job, and a Personal 
Development Manager who was assigned to help me make the most of my years
and nurture and develop my skill set. I feel that this system worked very 
well as it meant that there were always multiple avenues for dealing with
any problems that may have arisen from conflicts during my day job, and 
provided me with two equally useful but very different sets of advice. 
There were a few occasions when the dual management system added extra
complication to my role and caused confusion. For example, when I went
out on Giveback schemes, I had to seek approval from both managers. 
Nevertheless, I feel that the benefits of the system certainly outweighed 
the extra work in these cases.

I definitely noticed my changing identity as I progressed though the IBM
‘community of practice’ \cite{Communities}. Communities of practice are
groups of people who share a concern or a passion for something they do and
learn how to do it better as they interact regularly. At the beginning, I was
definitely a 'newcomer,' situated on the periphery and, although I was not
exactly an 'old timer' by the end of my thirteen months, I was certainly an
experienced 'practitioner' compared to those who had just begun their year with
IBM. During my year, learning was not simply a process of assimilation nor of
learning transfer but changed my very identity as mentors became colleagues and
then came to me to ask how to approach certain objectives. For instance, at the
beginning of the year as a 'newcomer' I struggled to understand how to use the 
mainframe terminal system and 
frequently sought advice from my team leader. Towards
the end of my time at IBM my advice regarding Java programming, specifically the
Eclipse Plug-in development, was highly-sought by my colleagues.

As I built up my bank of the 'tacit knowledge' \cite{SocialKnowing} of the IBM, 
community, I found that I
not only defined myself through what I could do with that knowledge, but I also
defined myself through what I no longer needed to do to acquire the necessary
knowledge. That is, I recognised the transition from someone who needed help   
to someone who provided help.

In summary, I am a very different person from when I left Aberystwyth University
in the second year of my studies. The value of my year in industry has not been
what it will add to my CV, nor in a healthy bank balance. The value has been in
the skills I have developed and connections I have made. I feel that these new
skills will help me both with my final year of studies and as an Information 
Technology professional moving forward as a new graduate.

\pagebreak

\bibliography{report}
%set style to plain
\bibliographystyle{plain}

\end{document}

