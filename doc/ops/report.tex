% Outline Project Specification Report

%The document is a report
\documentclass[12pt,a4paper]{article}

%define horizontal rule
\newcommand{\HRule}{\rule{\linewidth}{0.5mm}}

\usepackage{fullpage}

%use the listings package
\usepackage{listings}
%use the English language
\usepackage[english]{babel}
%use graphics
\usepackage{graphicx}
%use wrap figures
\usepackage{wrapfig}
%geometry stuffs
\usepackage{lscape}
%use natbib bibliography package
%\usepackage[numbers]{natbib}
%use harvard bibliography package
%\usepackage{harvard}	
%use captions
\usepackage{caption}
%use multi-row tables
\usepackage{multirow}
\usepackage{url}

\begin{document}

%use harvard citations
%\citationstyle{agsm}

%include the title page
\begin{titlepage}
 
\begin{center}
 
 
% Upper part of the page
\includegraphics[width=0.20\textwidth]{./Gerald_G_Man_in_Suit.png}\\[1cm]
 
\textsc{\LARGE Aberystwyth University}\\[1.5cm]
 
\textsc{\LARGE Industrial Year Report}\\[0.5cm]
 
 
% Title
\HRule \\[0.4cm]
{ \huge \bfseries Blue Harvest - My Year at IBM }\\[0.4cm]
 
\HRule \\[1.5cm]

 % Author and student ID
\begin{minipage}{0.4\textwidth}
\begin{flushleft} \large
\emph{Author:}\\
James \textsc{Ravenscroft}
\end{flushleft}
\end{minipage}
\begin{minipage}{0.4\textwidth}
\begin{flushright} \large
\emph{Student ID:} \\
090407039
\end{flushright}
\end{minipage}

\vfill
 
% Bottom of the page
{\large \today}
 
\end{center}
 
\end{titlepage}



%some definitions for paragraph layout stuff
\setlength{\parindent}{0pt}
\setlength{\parskip}{1.5ex plus 0.5ex minus 0.2ex}

%\tableofcontents

\pagebreak

\section{Project Description}

For scientists carrying out background research for a paper they are authoring, or looking for 
information on a specific topic that interests them, the vast quantity of
research papers available online can often be quite daunting, making 
choosing which papers to read very difficult. This problem
is exacerbated when there is a time limit on the research to be carried
out. Partridge aims to aid the reader in finding documents relative to their
interests through the use of Artificial Intelligence techniques. This should
make the research process more efficient and save the reader time. The program
may also help them find papers they may otherwise not have read.

Partridge will use a web interface to allow a user to indicate which
specific scientific fields they are interested in - either through manual entry
of interests and types of report or by allowing the application to collect a
list of papers that the user has already read and to learn their user profile.
The program will then suggest sets of papers from the corpus depending on their
contents and relevance to the user. The program will make these recommendations
based upon a machine learning approach to classifying the text corpus and
learning the user's preferences.

\section{Work to be tackled}
Whilst a web interface for the program will need to be implemented, the main
body of the work in this project comes from the backend of the system and the
Artificial Intelligence that drives it.

Firstly, there will need to be some significant research into Natural Language
Parsing techniques and how different features of the text can be used to
classify papers and help the system to understand the differences between them.

Once these features are understood, it will be possible to set up a learning
algorithm to analyse some collections of scientific papers and store
information about the differences between them. The system will make use of
some third party software to help with this process, however I will be using
knowledge from CS36110 Machine Learning as well as any relevant research papers
and text books to build a paper classifier myself.


\section{Project Deliverables}

\bibliographystyle{plain}
\nocite{*}
\bibliography{report}

\end{document}
