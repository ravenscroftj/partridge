%%%%%%%%%%%%%%%%%%%% author.tex %%%%%%%%%%%%%%%%%%%%%%%%%%%%%%%%%%%
%
% sample root file for your "contribution" to a contributed volume
%
% Use this file as a template for your own input.
%
%%%%%%%%%%%%%%%% Springer %%%%%%%%%%%%%%%%%%%%%%%%%%%%%%%%%%%%%%%%%


% RECOMMENDED %%%%%%%%%%%%%%%%%%%%%%%%%%%%%%%%%%%%%%%%%%%%%%%%%%%
\documentclass{svmult}
%
%% choose options for [] as required from the list
%% in the Reference Guide
%
\usepackage{mathptmx}       % selects Times Roman as basic font
\usepackage{helvet}         % selects Helvetica as sans-serif font
\usepackage{courier}        % selects Courier as typewriter font
\usepackage{type1cm}        % activate if the above 3 fonts are
                             % not available on your system
%
\usepackage{makeidx}         % allows index generation
\usepackage{graphicx}        % standard LaTeX graphics tool
%                             % when including figure files
\usepackage{multicol}        % used for the two-column index
\usepackage[bottom]{footmisc}% places footnotes at page bottom
%
%% see the list of further useful packages
%% in the Reference Guide
%
%\makeindex             % used for the subject index
%                       % please use the style svind.ist with
%                       % your makeindex program
%
%%%%%%%%%%%%%%%%%%%%%%%%%%%%%%%%%%%%%%%%%%%%%%%%%%%%%%%%%%%%%%%%%%%%%%%%%%%%%%%%%%%%%%%%%%
%
\begin{document}


\title*{Partridge: An effective system for the automatic classification of the types of academic papers}
\titlerunning{Partridge: classifying academic papers}
% Use \titlerunning{Short Title} for an abbreviated version of
% your contribution title if the original one is too long
\author{James Ravenscroft \and Maria Liakata \and Amanda Clare}
% Use \authorrunning{Short Title} for an abbreviated version of
% your contribution title if the original one is too long
\institute{James Ravenscroft, Amanda Clare \at Department of Computer Science, Aberystwyth University, Aberystwyth, SY23 3DB, UK  \email{jrr9@aber.ac.uk}
\and Maria Liakata \at Name, Address of Institute \email{name@email.address}}

%
% Use the package "url.sty" to avoid
% problems with special characters
% used in your e-mail or web address
%
\maketitle

\abstract{Partridge is a system that enables intelligent search for academic papers. Here we describe how Partridge implements the automatic classification of the type of a publication. For each paper, Partridge automatically extracts the full paper content from PDF files, determines sentence boundaries, automatically labels the sentences with core scientific concepts, and then uses a random forest model to classify the paper type. We show that the type of a paper can be reliably predicted by a model which analyses the distribution of core scientific concepts within the sentences of the paper. We discuss the appropriateness of many of the existing paper types used by major journals, and their corresponding distributions. Partridge is online and available for use, includes a browser-friendly bookmarklet for new paper submission, and demonstrates a range of possibilities for more intelligent search in the scientific literature.}

%\abstract*{Each chapter should be preceded by an abstract (10--15 lines long) that summarizes the content. The abstract will appear \textit{online} at \url{www.SpringerLink.com} and be available with unrestricted access. This allows unregistered users to read the abstract as a teaser for the complete chapter. As a general rule the abstracts will not appear in the printed version of your book unless it is the style of your particular book or that of the series to which your book belongs.
%Please use the 'starred' version of the new Springer \texttt{abstract} command for typesetting the text of the online abstracts (cf. source file of this chapter template \texttt{abstract}) and include them with the source files of your manuscript. Use the plain \texttt{abstract} command if the abstract is also to appear in the printed version of the book.}

%\abstract{Each chapter should be preceded by an abstract (10--15 lines long) that summarizes the content. The abstract will appear \textit{online} at \url{www.SpringerLink.com} and be available with unrestricted access. This allows unregistered users to read the abstract as a teaser for the complete chapter. As a general rule the abstracts will not appear in the printed version of your book unless it is the style of your particular book or that of the series to which your book belongs.\newline\indent
%Please use the 'starred' version of the new Springer \texttt{abstract} command for typesetting the text of the online abstracts (cf. source file of this chapter template \texttt{abstract}) and include them with the source files of your manuscript. Use the plain \texttt{abstract} command if the abstract is also to appear in the printed version of the book.}

\section{Introduction}
\label{sec:1}
Since the advent of the `Digital Age' and the ability of computers to copy and reproduce information for a negligible cost, the amount of information available to researchers has been increasing drastically.  B-C Bj\"{o}rk (2009) estimates that approximately 1.4 Million papers were published in the year 2006 alone\cite{bjork2009}.
As available information increases, relevant material becomes progressively more difficult to find manually and the need for an automated information retrieval tool more apparent.
There are already a large number of information retrieval and recommendation systems for scientific publications.  Many of these systems, such as AGRICOLA (\url{http://agricola.nal.usda.gov/}), the Cochrane Library(\url{http://www.thecochranelibrary.com/}) and Textpresso (\url{http://www.textpresso.org/}) index only publications from predefined journals or topics (for the above examples, Agriculture, Biology and Bioinformatics respectively).
Unfortunately, these domain specific indexing systems usually only contain a small subset of papers, excluding potentially crucial literature because it does not quite fit into the subject domain. 
%This problem is often exacerbated by the decision to leave out some papers and articles because the system administrators do not have permission from the author or publisher to include them. 
The value of these systems to their users is often restricted by the small proportion of available literature that they index, forcing researchers to use multiple, domain specific, search engines for their queries.
In contrast, there are also a number of interdisciplinary indexing systems and online journals such as arXiv( \url{http://arxiv.org/}), PloSOne(\url{http://plosone.org/}), and JSTOR (\url{http://www.jstor.org/}), that try to incorporate wide ranges of papers from as many disciplines as possible. The traits of these systems often complement those of their domain-specific counterparts; they provide a comprehensive collection of literature but insufficient filtering and indexing capabilities.
One of the most publicised and well known interdisciplinery scientific literature search systems is Google Scholar (\url{http://scholar.google.com}). Google offers advanced query options specific to Scholar that allow searching by author, year and for words that occur only in the document title.

However, the document title is just one part of the important structure to be found within a scientific document.
Liakata \emph{et al.} (2012) describe a system for automatically processing and classifying sentences in a research paper according to the core scientific concept (CoreSC) that they describe\cite{Liakata2012}. 
There are 13 CoreSCs, including {\em hypothesis}, {\it background}, {\em result}, {\em method} and {\em conclusion}.
CoreSC labels can be allocated to all sentences in a scientific paper in order to identify which scientific concept each sentence encapsulates.
SAPIENTA (\url{http://www.sapientaproject.com}) is a machine learning application which can automatically annotate all sentences in a scientific paper with their CoreSC labels. It was trained using a corpus of physical chemistry and biochemistry research papers whose sentences were manually annotated using the CoreSC\cite{LIAKATA10.644} scheme. 
An intelligent academic information retrieval system can use this information in order to provide better filtering and search capabilities for researchers. 
The ability to search for phrases and keywords by CoreSC will facilitate context-aware keyword search, that allows researchers to only accept papers where a term appears in sentences with a specific CoreSC label.
This can be used to greatly improve both the precision with which researchers are able to perform searches for scientific literature and the accuracy of those searches in terms of relevance to the reader.

The type of a paper ({\em review}, {\em case study}, {\em research}, {\em perspective}, etc) is another useful feature through which a user can narrow down the results of a search. 
The type of a paper can then be used to augment queries.
For example, a user may search for a {\em review} paper containing the keywords ``PCR microfluidics'', or a {\em research} paper with a {\em hypothesis} containing the keywords ``cerevisiae'' and ``glucose''.   
Such paper types are not yet standardised by journals.
We expect the structure of a paper to reflect its paper type. 
For example, review papers would be expected to contain a large amount of background material.
In this article, we describe the application of machine learning (using random forests) to create predictive models of a paper's type, using the distribution of CoreSC labels found in the full text of the paper. 

This model of paper type is currently in use in our Partridge system, which has been created as an intelligent full-text search platform for scientific papers. 
Partridge can make use of automatically derived CoreSC sentence labels and automatically derived paper types, to allow deeper information queries.  
We discuss the reliability of this model of paper types and the insights that have been gained for the authorship of papers.



%Use the template \emph{chapter.tex} together with the Springer document class SVMono (monograph-type books) or SVMult (edited books) to style the various elements of your chapter content in the Springer layout.

%Instead of simply listing headings of different levels we recommend to let every heading be followed by at least a short passage of text. Further on please use the \LaTeX\ automatism for all your cross-references and citations. And please note that the first line of text that follows a heading is not indented, whereas the first lines of all subsequent paragraphs are.

\section{Methods}
\label{sec:2}

Discuss in more detail how Partridge processes papers (PDFX, sentence splitting) 

how types were chosen,

how papers were acquired 

how the machine learning is conducted (random forest using orange, features, validation methods).

% Always give a unique label
% and use \ref{<label>} for cross-references
% and \cite{<label>} for bibliographic references
% use \sectionmark{}
% to alter or adjust the section heading in the running head


%Use the standard \verb|equation| environment to typeset your equations, e.g.
%
%\begin{equation}
%a \times b = c\;,
%\end{equation}
%
%however, for multiline equations we recommend to use the \verb|eqnarray|
%environment\footnote{In physics texts please activate the class option
%\texttt{vecphys} to depict your vectors in \textbf{\itshape
%boldface-italic} type - as is customary for a wide range of physical
%subjects}.
%\begin{eqnarray}
%a \times b = c \nonumber\\
%\vec{a} \cdot \vec{b}=\vec{c}
%\label{eq:01}
%\end{eqnarray}

%\subsection{Subsection Heading}
%\label{subsec:2}
%Instead of simply listing headings of different levels we recommend to let every heading be followed by at least a short passage of text. Further on please use the \LaTeX\ automatism for all your cross-references\index{cross-references} and citations\index{citations} as has already been described in Sect.~\ref{sec:2}.

%\begin{quotation}
%Please do not use quotation marks when quoting texts! Simply use the \verb|quotation| environment -- it will automatically render Springer's preferred layout.
%\end{quotation}


%\subsubsection{Subsubsection Heading}
%Instead of simply listing headings of different levels we recommend to let every heading be followed by at least a short passage of text. Further on please use the \LaTeX\ automatism for all your cross-references and citations as has already been described in Sect.~\ref{subsec:2}, see also Fig.~\ref{fig:1}\footnote{If you copy text passages, figures, or tables from other works, you must obtain \textit{permission} from the copyright holder (usually the original publisher). Please enclose the signed permission with the manucript. The sources\index{permission to print} must be acknowledged either in the captions, as footnotes or in a separate section of the book.}




%For typesetting numbered lists we recommend to use the \verb|enumerate| environment -- it will automatically render Springer's preferred layout.

%\begin{enumerate}
%\item{Livelihood and survival mobility are oftentimes coutcomes of uneven socioeconomic development.}
%\begin{enumerate}
%\item{Livelihood and survival mobility are oftentimes coutcomes of uneven socioeconomic development.}
%\item{Livelihood and survival mobility are oftentimes coutcomes of uneven socioeconomic development.}
%\end{enumerate}
%\item{Livelihood and survival mobility are oftentimes coutcomes of uneven socioeconomic development.}
%\end{enumerate}



%For unnumbered list we recommend to use the \verb|itemize| environment -- it will automatically render Springer's preferred layout.

%\begin{itemize}
%\item{Livelihood and survival mobility are oftentimes coutcomes of uneven socioeconomic development, cf. Table~\ref{tab:1}.}
%\begin{itemize}
%\item{Livelihood and survival mobility are oftentimes coutcomes of uneven socioeconomic development.}
%\item{Livelihood and survival mobility are oftentimes coutcomes of uneven socioeconomic development.}
%\end{itemize}
%\item{Livelihood and survival mobility are oftentimes coutcomes of uneven socioeconomic development.}
%\end{itemize}

%\begin{figure}[t]
%\sidecaption[t]
% Use the relevant command for your figure-insertion program
% to insert the figure file.
% For example, with the option graphics use
%\includegraphics[scale=.65]{figure}
%
% If no graphics program available, insert a blank space i.e. use
%\picplace{5cm}{2cm} % Give the correct figure height and width in cm
%
%\caption{Please write your figure caption here}
%\caption{If the width of the figure is less than 7.8 cm use the \texttt{sidecapion} command to flush the caption on the left side of the page. If the figure is positioned at the top of the page, align the sidecaption with the top of the figure -- to achieve this you simply need to use the optional argument \texttt{[t]} with the \texttt{sidecaption} command}
%\label{fig:2}       % Give a unique label
%\end{figure}

%\runinhead{Run-in Heading Boldface Version} Use the \LaTeX\ automatism for all your cross-references and citations as has already been described in Sect.~\ref{sec:2}.

%\subruninhead{Run-in Heading Italic Version} Use the \LaTeX\ automatism for all your cross-refer\-ences and citations as has already been described in Sect.~\ref{sec:2}\index{paragraph}.


% Use the \index{} command to code your index words
%
% For tables use
%
%\begin{table}
%\caption{Please write your table caption here}
%\label{tab:1}       % Give a unique label
%
% Follow this input for your own table layout
%
%\begin{tabular}{p{2cm}p{2.4cm}p{2cm}p{4.9cm}}
%\hline\noalign{\smallskip}
%Classes & Subclass & Length & Action Mechanism  \\
%\noalign{\smallskip}\svhline\noalign{\smallskip}
%Translation & mRNA$^a$  & 22 (19--25) & Translation repression, mRNA cleavage\\
%Translation & mRNA cleavage & 21 & mRNA cleavage\\
%Translation & mRNA  & 21--22 & mRNA cleavage\\
%Translation & mRNA  & 24--26 & Histone and DNA Modification\\
%\noalign{\smallskip}\hline\noalign{\smallskip}
%\end{tabular}
%$^a$ Table foot note (with superscript)
%\end{table}
%


\section{Results and Discussion}
\label{sec:3}

Present the random forest results. Maybe also the clustering results. Include a few pie charts showing example distributions for typical papers from a few types. 

Discuss why it's possible to predict type from CoreSC, which classes are confused and why, which types are well defined by CoreSC. 

%\begin{description}[Type 1]
%\item[Type 1]{That addresses central themes pertainng to migration, health, and disease. In Sect.~\ref{sec:1}, Wilson discusses the role of human migration in infectious disease distributions and patterns.}
%\item[Type 2]{That addresses central themes pertainng to migration, health, and disease. In Sect.~\ref{subsec:2}, Wilson discusses the role of human migration in infectious disease distributions and patterns.}
%\end{description}


%\begin{svgraybox}
%If you want to emphasize complete paragraphs of texts we recommend to use the newly defined Springer class option \verb|graybox| and the newly defined environment \verb|svgraybox|. This will produce a 15 percent screened box 'behind' your text.

%If you want to emphasize complete paragraphs of texts we recommend to use the newly defined Springer class option and environment \verb|svgraybox|. This will produce a 15 percent screened box 'behind' your text.
%\end{svgraybox}



%\begin{theorem}
%Theorem text goes here.
%\end{theorem}
%
% or
%
%\begin{definition}
%Definition text goes here.
%\end{definition}

%\begin{proof}
%\smartqed
%Proof text goes here.
%\qed
%\end{proof}

%\paragraph{Paragraph Heading} %
%Instead of simply listing headings of different levels we recommend to let every heading be followed by at least a short passage of text. Further on please use the \LaTeX\ automatism for all your cross-references and citations as has already been described in Sect.~\ref{sec:2}.

%Note that the first line of text that follows a heading is not indented, whereas the first lines of all subsequent paragraphs are.
%
% For built-in environments use
%
%\begin{theorem}
%Theorem text goes here.
%\end{theorem}
%
%\begin{definition}
%Definition text goes here.
%\end{definition}
%
%\begin{proof}
%\smartqed
%Proof text goes here.
%\qed
%\end{proof}
%


\section{Conclusion}
\label{sec:4}

Summary of findings.

Discuss the potential for intelligent paper search opportunities in general and the future for Partridge.

\begin{acknowledgement}
Thanks to the Leverhulme for Maria's funding....
\end{acknowledgement}
%
%\section*{Appendix}
%\addcontentsline{toc}{section}{Appendix}
%%
%%
%When placed at the end of a chapter or contribution (as opposed to at the end of the book), the numbering of tables, figures, and equations in the appendix section continues on from that in the main text. Hence please \textit{do not} use the \verb|appendix| command when writing an appendix at the end of your chapter or contribution. If there is only one the appendix is designated ``Appendix'', or ``Appendix 1'', or ``Appendix 2'', etc. if there is more than one.

%\begin{equation}
%a \times b = c
%\end{equation}

\bibliographystyle{plain}
\bibliography{partridge}

%\input{referenc}
\end{document}
