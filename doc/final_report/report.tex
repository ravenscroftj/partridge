% Final Dissertation Report

%The document is a report
\documentclass[11pt,a4paper,oneside]{book}

%define horizontal rule
\newcommand{\HRule}{\rule{\linewidth}{0.5mm}}

\usepackage{fullpage}
\usepackage{pdfpages}
%use the listings package
\usepackage{listings}
%use the English language
\usepackage[english]{babel}
%use graphics
\usepackage{graphicx}
%use wrap figures
\usepackage{wrapfig}
%geometry stuffs
\usepackage{lscape}
%use captions
\usepackage{caption}
%use multi-row tables
\usepackage{multirow}
\usepackage{url}
\usepackage{breakurl}
\usepackage{subcaption}
\usepackage{rotating}
\usepackage{pdflscape}

\usepackage[nottoc,numbib]{tocbibind}

\usepackage{fancyhdr}
\setlength{\headheight}{15pt}

\pagestyle{fancy}



\begin{document}

%include the title page
\newcommand{\Revision}{76c5729}

\begin{titlepage}
 
\begin{center}
 
 
% Upper part of the page
\includegraphics[width=0.20\textwidth]{./Gerald_G_Man_in_Suit.png}\\[1cm]
 
\textsc{\LARGE Aberystwyth University}\\[1.5cm]
 
\textsc{\LARGE Industrial Year Report}\\[0.5cm]
 
 
% Title
\HRule \\[0.4cm]
{ \huge \bfseries Blue Harvest - My Year at IBM }\\[0.4cm]
 
\HRule \\[1.5cm]

 % Author and student ID
\begin{minipage}{0.4\textwidth}
\begin{flushleft} \large
\emph{Author:}\\
James \textsc{Ravenscroft}
\end{flushleft}
\end{minipage}
\begin{minipage}{0.4\textwidth}
\begin{flushright} \large
\emph{Student ID:} \\
090407039
\end{flushright}
\end{minipage}

\vfill
 
% Bottom of the page
{\large \today}
 
\end{center}
 
\end{titlepage}



%some definitions for paragraph layout stuff
\setlength{\parindent}{0pt}
\setlength{\parskip}{1.5ex plus 0.5ex minus 0.2ex}



\thispagestyle{plain}

\begin{center}
\textsc{\large Statement of Originality}
\end{center}

This submission is my own work, except where clearly indicated.  

I understand that there are severe penalties for plagiarism and other unfair
practice, which can lead to loss of marks or even the withholding of a degree.
I have read the sections on unfair practice in the Students’ Examinations
Handbook and the relevant sections of the current Student Handbook of the
Department of Computer Science.  

I understand and agree to abide by the University’s regulations governing these
issues.\\\\

........................................

\pagebreak
\thispagestyle{plain}

\begin{center}
\textsc{\large Acknowledgements}
\end{center}

I would like to take this opportunity to thank both of my
supervisors, Dr Amanda Clare and Dr Maria Liakata, for giving up so much of their
time to support me in designing and developing this project. 

Thanks to the users who tested Partridge once it was made generally available
and for their feedback in the form of bug reports and suggestions.

Finally, I am eternally grateful for the support of my family and friends who
have put up with my gripes and moans throughout the duration of this project
and, without whom, I couldn't have achieved such an ambitious project.

\pagebreak
\thispagestyle{plain}


\begin{center}
\textsc{\large Abstract}
\end{center}

When faced with the great number of scientific papers made general available on
the internet, it is often very difficult for researchers to find relevant
scientific literature. This project aimed to build an information retrieval and
recommendation tool that uses Natural Language Processing and Machine Learning
techniques to present the most relevant scientific papers to a user based upon
their search criteria and preferences.

A preprocessing system that exploits existing Natural Language Processing tools
was used to analyse a large sample of research papers, labelling the presence
of conceptual zones within each paper and extracting metadata.  Decision trees
were also trained to classify papers into well established type categories
based upon the relative proportions of the types of conceptual zones
present within the papers. 

A web interface was built to facilitate fine grain search, making use of the
zone information captured during preprocessing. Filtering of papers using their
type as determined by the preprocessor was also built in. Individual papers
returned during searches are presented in their own profile page that provides
charts and extra metadata to help researchers to understand the paper's
content. A page allowing authors and rightsholders to submit their own papers
for analysis was also created to allow the scientific community to extend the
impact and utility of the system.

The machine learning systems used within the project have been evaluated and
found to be accurate to 87.9\% when classifying a paper's type. The possibility
of extending the machine learning capabilities of the system to categorise
papers by subject domain and outcome of experimental results is discussed as
well as the potential for further improvements to the user interface.

A working version of the software has been made generally available at
\url{http://farnsworth.papro.org.uk} and can be explored and used by the
general public. It is hoped that over the coming months, the system will grow
to become a far reaching, widely used tool for academic research.


\tableofcontents

\pagebreak

\mainmatter

\chapter{Introduction}
%%%%------------------------------------------------
%
%  Include for chapter one of dissertation: introduction
%
%%%%------------------------------------------------
\subsection{Context}

\subsection{Background}

\subsection{Objectives}

\subsection{Literature Review}


\chapter{ Development Methodology }
\subsection{Development Methodology}

\subsubsection{Existing Methodologies}
Selecting a suitable development methodology for building Partridge is another
very important choice for the project.

Under the traditional `Waterfall' Software Development model, Requirements
Gathering, Analysis, Program Design, Coding, Testing and Operations were all
defined as formal phases in the development cycle. There is little flexibility
other than moving back up the waterfall to rectify mistakes after
testing\cite{Royce:1987:MDL:41765.41801}. This model was very focused on
paperwork and bureaucracy, trying to maintain a paper trail and manage risk
through accountability \emph{(Ibed)}. This approach to software development is
very heavyweight and slow and often produced software that did not match the
users' needs as a result \cite{Boehm1988}.

As an alternative to the heavyweight Waterfall approach, Beck \emph{et al.} came up
with the principle of the Agile Manifesto, favouring a lightweight, responsive
development model over the heavyweight slow waterfall
system\cite{beck2001agile}. Many of Beck's ideas focus around working in a team
of developers and prioriting communication between team members \emph{(Ibid.)}.
This is most prominent in the Extreme Programming (XP) method of software
development. Since Partridge is an individual project, XP is not really
applicable. However, some concepts like rapid prototyping/spike work and
iterative release cycles will be used as part of the Partridge development
methodology.

\subsubsection{ Partridge's Development Methodology}

Partridge is an individual project but does involve discussions with
supervisors. The customers have been identified as the end-users of the system.
Therefore, a customised methodology has been adopted. Firstly, all design and
planning documentation have been written up and placed on a wiki which is
accessible and modifiable by the author and both supervisors. This creates a
paper trail for all tasks and also allows collaboration between involved
parties through the Internet. A full printout of the wiki is available in
Appendix \ref{sec:wiki}. 

Weekly meetings are held with both supervisors. The notes from the preceeding
week are analysed and each task discussed in depth. New tasks are then noted
down along with any observations that should be documented. These new notes are
uploaded to the wiki the following day or earlier. Each party present at the
meeting adds their own observations to the notes page. This page is then
reviewed at the next meeting. As seen in Appendix \ref{sec:wiki}, this practice
has already been running for several weeks and has so far proven to be highly
effective.

Partridge will adopt an Agile approach to release cycles, producing a working
software package at iterations of one month.  Each iteration, the software will
include more of the desired functionality discussed above and in the wiki.
GitHub's issue manager program is being used to track tasks and bugfixes and
plan which tasks will be carried out in which iteration. Tasks that are created
in a full iteration (where no development time is left) will be added to a
backlog and integrated in the next iteration with enough development time to
contain it. Tasks are also assigned a priority, higher priority issues being
tackled before low priority ones.

Partridge's testing strategy consists of multiple unit tests that are run at
integration of new code into the codebase. As soon as the first release is
built, Partridge will be made available for use by the public and users
encouraged to test the system and submit any bugs via the GitHub issue manager.
It is hoped that colleagues at Aberystwyth University and Dr. Liakata's
colleages at the EBI will try to use the system one it becomes available.

\subsubsection{ Work Timeline }

The tasks involved in Partridge have been carefully calculated and prioritised.
They were then added to the GitHub issue management system and a report
generated listing them in the order that they are expected to be
accomplished. This report can be seen in Appendix \ref{sec:timeline}. 


\subsubsection{ Mid-Project Demonstration Plan}

The mid-project demonstration is scheduled for after iteration two of the
Partridge project. If everything runs to schedule, then at this point it will
be possible to demonstrate keyword search within the project's database and
filter based upon the polarity of the paper's results. For redundancy purposes,
Partridge will be configured to run as a server on multiple computers. Both
this and the final project demonstration will require a room with Internet
access. However, should this be unavailable, then Partridge could be run
locally on the author's laptop.

\subsubsection{ Final Project Demonstration Plan}

The final demonstration of Partridge will be fairly similar to the Mid-Project
demonstration. However, it should include all of the planned classifiers and if
there is extra time on the project, the profiling/recommendation engine will
also be demonstrated. This demonstration will use the same redundancy
precautions as the Mid-project demonstration above, and will also need a room
with the internet if available.


\chapter{Design}
%%-----------------------------------------------
%
% Include for design chapter of dissertation
%
%%----------------------------------------------



\section{ Target Platform \& High Level Considerations }

\subsection{Use Case Analysis}

Given the project objectives identified above, four primary use cases were
identified for Partridge as shown in Figure \ref{fig:use_cases}. Users can be
identified as researchers and people interested in finding and reading
scientific papers that interested them. This would involve querying the
database for relevant papers, viewing metrics on the papers that have been used
to classify them and also downloading the original paper for reading. A subset
of users were identified as authors who might be interested in adding their own
papers to the Partridge instance.

\begin{figure}[!h]
\centering
\includegraphics[width=0.4\textwidth]{images/design/use_cases.png}
\caption{Use Case Diagram for Partridge Project}
\label{fig:use_cases}
\end{figure}

\subsection{ Target Platform}

It was decided from the outset that Partridge would be a web-based tool.
Web-based systems can generally be run on any computer with access to the
internet and a modern web browser. This also means that there is no need for
the end user to install or configure extra software on their computers, making
Partridge accessible to non-technical users and those who have aggressive
software restrictions on their computers, such as GPs and users of public
computers in libraries or on academic sites.

In order to function as a website, Partridge needed to be developed as a server
application that produced Hyper Text Markup Language (HTML) output that a web
browser could interpret and display. The browser rendering the forms must then
communicate with a backend system capable of querying and returning papers to
the user based on their input. This lead to the decision to split development
of Partridge into two core components: the Web Frontend comprising of user
interface elements and presentation logic and the Web Backend comprising of
intelligent systems to classify, filter and query papers using some of the NLP
techniques discussed previously.

One of the main requirements for a web server is that it responds to requests
relatively quickly, ideally within 10 seconds or less. If they don't, the user
may get impatient and give up trying to use the system, or in some cases, their
web browser may timeout and stop trying to load the page. Many of the processes
involved in extracting meaningful CoreSC information from papers are very slow
slow. This means that processing papers during web requests is not really
feasible, since it may take several minutes for a single paper to be annotated
and classified. Therefore, a third core component was identified: a paper
preprocessor service that runs on the server and converts, annotates and
classifies papers as soon as they are uploaded.

\subsection{ Programming and Development Environment }

Rather than building a web server from the ground up, most modern web
applications are written in higher level languages that run on top of standard,
open source web servers such as Apache or nginx. Common language choices
include PHP, Python, Perl and Java, all of these languages supported by large
communities of developers and users who provide with excellent documentation
and support if required.  For the implementation of Partridge, Python was
selected as the programming language of choice. This was partially due to the
author's familiarity with the language and also due to the availability of
stable data mining\cite{curk05} and natural language
processing\cite{bird2009natural} libraries for the Python programming
environment. Python is an interpreted cross-platform language, minimising
deployment problems and supports natively compiled C and C++ extension
libraries allowing intensive processing  and number crunching to be offloaded
to native plugins, increasing application processing speed.

To further increase development speed, a Python framework for developing web
applications called \emph{Flask\cite{flask2012}} would be used to build the web
backend.

The development of the application would be carried out on a Linux desktop
computer. However, the portable nature of Python, Apache and the required
libraries meant that Partridge could theoretically be deployed to Windows and
Mac computer systems if required.

The Web Frontend for the system would need to be written using a combination of
HTML markup and JavaScript. HTML is not a programming language in itself and
doesn't contain any logic. It is merely a code representation of the interface
to be rendered in the browser window which is interpreted once it has been
downloaded from the web server. In order to carry out actions on the HTML such
as validating user input and manipulating parts of the display, JavaScript is
also downloaded from the web server and interpreted by the browser. Although
this requires the developer to have knowledge of two extra technologies on top
of Python, it also makes enforcing the design principle of keeping presentation
and logic separate a lot easier.

Due to the large number of popular web browsers on the market, implementing
JavaScript in a slightly different way, Javascript can be very difficult to
write in a cross-browser compatible way.  To maximise compatibility and reduce
development time, it was decided that libraries such as jQuery which are
designed to carry out HTML manipulation behaviours in a uniform way cross all
compatible browsers.  As of February 2013, it is estimated that Chrome and
Firefox are currently the most popular web browsers, holding 50\% and 29.6\% of
the market share respectively\cite{browserstats2013}. Internet Explorer is the
third most popular browser. However, there are several well known compatibility
issues with Internet Explorer, Microsoft themselves condemning version 6.0 of
their browser\cite{ie6death}. For this reason, it was decided that Partridge
would only support Firefox and Chrome. A working interface in modern editions
of the Internet Explorer family of browsers would be a bonus.  However, no time
was allocated to ensuring compatibility of Partridge with Internet Explorer.

\section{ System Architecture }


\begin{figure}[!ht]
\center
\includegraphics[width=0.9\textwidth]{images/design/components_high_level.png}
\caption{High Level Component Layout for Partridge}
\label{fig:high_level_components}
\end{figure}

Figure \ref{fig:high_level_components} shows a very high level description of
the system's three main components, the database server and the ways that they
communicate. Communication between the Web Interface and the other two
components is carried out via HTTP requests to the underlying Apache server.
Communication between the web backend and the paper preprocessor is implicit
via the data storage system. Direct communication between these components is
redundant since the web backend cannot gain any extra information from the
paper preprocessor while a paper is still being analysed and all finished
papers are added to the database which the web backend has full access to
anyway.

\subsection{ Data Storage }

\subsection{ Preprocessor }

The preprocessor module is used to annotate and analyse new papers once they 
have been uploaded by the user via the web interface. The preprocessor is designed 
as a daemon which uses pyinotify to hook into the operating system and monitor a directory 
for new papers that have been uploaded. As new papers are detected within the upload directory, 
they are placed into a process queue and the preprocessor analyses them on a first in first out basis.

When a paper is taken from the queue, a series of processes are run upon it to prepare 
it for storage in the database. 

\subsubsection{Conversion}

\subsubsection{Sentence Splitting}

\subsubsection{SAPIENTA Annotation}

\subsubsection{Type Analysis}


\subsection{ Web Backend }

\subsection{ Web Frontend }


\chapter{Implementation}
%%------------------------------------------------------
%  
%  Implementation include for dissertation
%
%------------------------------------------------------


\subsection{ Data Mining \& Paper Processing}


\subsection{ Web Interface \& Database System}


\subsection{ Paper Type Classification }


\chapter{Testing}
\label{chapter:testing}
%%------------------------------------------------------
%  
%  Testing include for dissertation
%
%------------------------------------------------------

%introduction to testing
Throughout the implementation of the project, verifying that systems provide
reliable, expected outputs for given inputs has remained a very important part
of the overall development process necessary to Partridge's development.  A
number of testing techniques have been used in order to evaluate the individual
behaviours of Partridge's key components, both separately and the system as a
whole. This chapter discusses the techniques used to test Partridge, the
results of these tests and any implications that these results have for further
development of the project. Combining all of these testing and evaluation
techniques provides an in-depth view of how useful Partridge is and how it could
be improved.

\section{ Unit Testing }

At the beginning of the project, Unit tests for Patridge's server and paper
preprocessing components were written in order to test that the main application worked before it was deployed to the production system at
\url{http://farnsworth.papro.org.uk}. These tests were maintained
throughout the project and occasionally had to be changed as the code was
refactored. As each new feature was added to the system, the whole suite of
tests was executed to provide integration coverage and make sure that new
features did not break existing code. 

\subsection{ Backend \& Infrastructure Testing}

The backend of the system was tested using Python unit tests based upon the
built in \emph{unittest} module and a third party testing library called
\emph{Nose} which provided automated test discovery. This meant that a user who
ran the command \emph{nose} from the project directory could automatically find
and run all of the unit tests in the project without having to set up
boilerplate test suites as in \emph{JUnit}.

In cases where Partridge calls remote services, reuses code that has already
been tested, or calls code that runs heavy computation, it can be inconvenient
or unhelpful to run unit tests that trigger these behaviours. To save time,
\emph{Mock}\cite{mock2013} was used for verifying these behaviours were being
called correctly instead.  \emph{Mock} allows the arbitrary replacement of
Python routines and objects with `mock' objects that mimic the behaviour of the
original routine and store information on how they were called or used. This
meant that calls to long-running routines and functions that already have tests
can be validated and a fake return value sent back to the caller without
having to actually execute the code. This technique also allows the isolation
of problems within the caller code and its commmunication with the target
functions and will pass even if the target code is broken provided that the
caller code is correct\cite{saff2004mock}. The Mock testing technique was used
heavily in the paper processor which calls SAPIENTA and PDFX and also wraps
several data classes which have their own unit tests.

Partridge's web views were tested using \emph{Flask}'s in-built test client.
This is an object attached to the \emph{Flask} app that can be used to emulate
a call to a particular view\cite{flask2012}. Cookies and form information can also be
manipulated through this client, enabling comprehensive testing of each of the backends
for Partridge's query, paper upload and bookmarklet capabilities.


\subsection{ Selenium \& Front-end Testing }

The frontend parts of Partridge that were written in Javascript and HTML5 also
had to be tested to verify that the query and upload forms behaved as expected.
Selenium\cite{seleniumide} was selected to facilitate these tests.

Selenium is a browser-based unit testing framework that allows developers to
script actions that a user might undertake whilst browsing a website. Tests for
selenium can be written in Firefox using the Selenium IDE framework. These
tests can then be exported as HTML files with embedded javascript instructions
and executed in any other browser to test compatibility with the product under
test.

During the development of Partridge, several suites of tests were written using
Selenium to verify that the query form, paper upload form and bookmarklet code
behave as expected. Not only do these tests verify the behaviour of the HTML
and javascript interface, but they also helped to check the higher level
functionality of Partridge. For example, the test suite for the Partridge query
form also helps to verify that query backend is working correctly. The Selenium
tests were executed as part of the integration process when a new feature was
added to Partridge.



\section{ Evaluating Learning Algorithms }
\label{sec:evaluation_learners}

As discussed above, Patridge uses a supervised learning algorithm, a Random
Decision Forest, to classify the type of each of the scientific papers added to
its database. Unlike traditional procedural algorithms that can be unit tested
and in some cases formally proven to behave uniformly at
runtime\cite{filliatre2007formal}, supervised learning algorithms are imperfect
systems that build `working models' from known data. This makes the performance
of supervised learning systems much more difficult to assess. Russell and
Norvig(2010) suggest that ``...a learning system is good if it produces hypotheses
that do a good job of predicting the classification of unseen
examples\cite{russell2010artificial}." It is suggested that when training a
supervised learning system, some of the example data should be kept back when
during the training phase and instead, used as `...unseen examples' and classified
by the newly trained system to judge its accuracy
\cite{alpaydin2004introduction}\cite{russell2010artificial}.

A common technique for partitioning data into `seen' and `unseen' or training
and test data sets is to simply split the data into two sets through the use of
random sampling.  A 60:40 divide between training and test data is common.
Witten explains that ``...it is better to use more than half the data for
training, even at the expense of test data\cite{witten2005data}." 

\subsection{ Three Fold Validation }

Three fold validation is a commonly used method for using all of the available
data for both testing and training whilst still maintaining the separation
between training and testing data. As the name suggests, test data is split
into three disjoint sets and then the classifier is trained on two of the sets
and tested with the third. This is repeated three times so that all sets are
used for both training and testing. Results from each of the folds can be
represented separately or an average taken.

\subsection{ Evaluation Result Analysis } 

Results from the evaluation methods discussed above can be presented in a
number of ways. When working with machine learning systems, it is important to
analyse performance using a selection of techniques in order to get a
comprehensive image of the capabilities and restrictions of the system.

Classifier Accuracy (CA), Receiver Operating Characteristics
(ROC) charts and Area Under the Curve (AUC) give a general overview of the
classifier's behaviour, whilst confusion matrices, recall, precision and
f-measure give a detailed view of which specific data types were most
frequently misclassified and what they were mistaken for. 

\subsection{ Classifier Accuracy } 

The CA of a supervised classifier is simply represented as the percentage of
the test data that the classifier was able to predict correctly. CA values that
tend towards 1.0 correspond to a more accurate classifier. In a Three Fold
validation, the CA for the classifier is calculated by taking the mean of the
individual performance values from each of the test/train cycles. 

3 Fold Validation on the Partridge corpus using the Random Forest Classifier
for paper type gives a CA measure of 87.9\%. While this provides a positive,
general overview of the classifier performance. It does not provide detail on
what data was incorrectly classifier nor any insight on how the system could be
improved. The following sections discuss techniques that provide more useful
measures of performance for machine learning classifiers.

\subsection{ Receiver Operating Characteristics Chart}

Receiver Operating Characteristics (ROC) Charts are used to visually represent
the trade-off between True Positives(TP) and False Positives(FP) as identified
by the classifier under evaluation. Witten (2005) explains that ROC is a method
reappropriated from signal processing analysts who use it to visualise the ``...hit
rate and false alarm rate over noisy communication
channels\cite{witten2005data}." The ratio of TP to FP classifications is
measured with increasing numbers of test samples from left to right. 

ROC charts are often divided into halves diagonally by a line indicating where
the TP-FP ratio is equal. There may be multiple plots on the axes if several
learning algorithms are under evaluation. A graph plot that presents a steep
curve that mostly fits into the upper left half of the graph shows a high TP-FP
ratio which is a sign that the classifier is working well.  Figure
\ref{fig:roc_random_forest} shows the ROC Chart for Partridge's Paper Type
Classifier after training on 60\% of the Partridge corpus and testing on 40\%.

\begin{figure}[!h]
\begin{center}
\includegraphics[width=0.65\textwidth]{images/testing/ROC.png}
\caption{ ROC Curve for Partridge's Random Forest Topic Classifier}
\label{fig:roc_random_forest}
\end{center}
\end{figure}

\subsection{ Area Under Curve }

The Area Under the Curve (AUC) measure is calculated as a percentage of  the
area underneath a ROC curve. This can again be used to give an indication of
the trade-off between true positive and false positive classification. It is a
percentage measurement, where 100\% indicates a high TP-FP ratio. AUC is often
used instead of visualising a ROC graph for brevity or where average
performance of a classifier over several train-test cycles is to be calculated.
The average AUC for Partridge's Random Tree Paper Type classifier is
96.6\% after three-fold validation is executed over the whole set of papers.

A comparison of this result with the above graph does show discrepancies between
simple 60:40 proportional validation and 3-fold validation; the average
performance of the latter seems to be slightly more accurate. This may be
because a slightly higher number of training examples (66\% rather than 60\%)
were used to train the classifiers during 3-fold validation and that the errors
were averaged out at the end of the process.

\subsection{ Confusion Matrices}

Confusion Matrices are a way of comprehensively showing which classes a machine
learning system struggles to discriminate between the most. Known classes are
shown on the vertical axis of the matrix and predicted classes are shown across
the horizontal. This allows inference of which data classes a trained
classifier struggles to label most frequently and any classes that it commonly
confuses them with. A set of confusion matrices for the Paper Type Classifier
in Partridge can be seen in Figure \ref{fig:conf_matrices} below.

\begin{figure}[!h]

\centering

\begin{subfigure}[b]{\textwidth}

\caption{Fold 1 Results}
\centering
\begin{tabular}{ | r l l l l |}
\hline
\multirow{5}{*}{\begin{sideways}{Actual Class}\end{sideways}}
&&\multicolumn{3}{c|}{Predicted Class} \\
&& \multicolumn{1}{c}{Case Study} &	Research&	\multicolumn{1}{c|}{Review}\\
\cline{3-5}
& \multicolumn{1}{c|}{Case Study} &	83&		2&		12\\
&\multicolumn{1}{c|}{Research}&	3&		255&		0\\
&\multicolumn{1}{c|}{Review}&	4&		2&		159\\
\hline

\end{tabular}
\end{subfigure}


\begin{subfigure}[b]{\textwidth}

\caption{Fold 2 Results}
\centering
\begin{tabular}{ | r l l l l |}
\hline
\multirow{5}{*}{\begin{sideways}{Actual Class}\end{sideways}}
&&\multicolumn{3}{c|}{Predicted Class} \\
&& \multicolumn{1}{c}{Case Study} &	Research&	\multicolumn{1}{c|}{Review}\\
\cline{3-5}
&\multicolumn{1}{c|}{Case Study}&   85    &         2&		10\\
&\multicolumn{1}{c|}{Research}&	0   &		256&		2\\
&\multicolumn{1}{c|}{Review}&	5&		1&		159\\
\hline
\end{tabular}

\end{subfigure}

\begin{subfigure}[b]{\textwidth}

\caption{Fold 3 Results}
\centering
\begin{tabular}{ | r l l l l |}
\hline
\multirow{5}{*}{\begin{sideways}{Actual Class}\end{sideways}}
&&\multicolumn{3}{c|}{Predicted Class} \\
&& \multicolumn{1}{c}{Case Study} &	Research&	\multicolumn{1}{c|}{Review}\\
\cline{3-5}
&\multicolumn{1}{c|}{Case Study}&	82&		3	&	12 \\
&\multicolumn{1}{c|}{Research}&	0&		257	&	1\\
&\multicolumn{1}{c|}{Review}&		7&		2	&	156\\
\hline
\end{tabular}

\end{subfigure}

\caption{ Confusion matricies for the 3-Fold validation process}
\label{fig:conf_matrices}

\end{figure}

These results show that Partridge's decision tree is very effective at
classifying the three document types in its database. It is also clear, however,
that whilst the classifier is outstanding at recognising research papers, it
sometimes struggles to discriminate between case study and review papers. This
may be because case studies can be considered a specific type of review paper
and the two classes share more similar CoreSC features than they do with
research papers.  

\subsection{ Precision, Recall and F-Measure }

From a classifier's confusion matrices, it is possible to calculate its
Precision, Recall and F-Measure values. These can be used to make further
assertions about the validity of a classifier's behaviour
\cite{witten2005data}.

The Precision of a classifier is the fraction of data samples retrieved that
were correctly classified. This is defined as:
\[Precision = \frac{tp}{tp+fp} \]

The Recall of a classifier is the proportion of samples within a given class
that the classifier is able to correctly classify. For example: the number of
apples that a classifier identifies as apples within a training set of fruit.
This is defined as:
\[Recall = \frac{tp}{tp+fn} \]

The F-Measure of a classifier is used to characterise both Recall and Precision
in a single value. The F-Measure of a classifer can be defined as:
\[
F-Measure = \frac{ 2 \times Recall \times Precision} { Recall + Precision } = 
\frac{2 . TP}{ 2 . TP + FP + FN }
\]


\begin{figure}[!th]

\centering
\begin{tabular}{| l | l | l | l | l |}
\hline
&        &\textbf{Recall}&\textbf{Precision}&\textbf{F-Measure}\\
\hline
\hline
\multirow{4}{*}{Fold 1} & Case Study & 0.85 & 0.92 & 0.89 \\
                        & Review     & 0.96 & 0.92 & 0.94 \\
                        & Research   & 0.98 & 0.98 & 0.98 \\
\hline

\multirow{4}{*}{Fold 2} & Case Study & 0.87 & 0.94 & 0.91 \\
                        & Review     & 0.96 & 0.92 & 0.94 \\
                        & Research   & 0.99 & 0.98 & 0.99 \\

\hline

\multirow{4}{*}{Fold 3} &Case Study & 0.84 & 0.92 & 0.88 \\
                        & Review    & 0.95 & 0.92 & 0.93 \\
                        & Research  & 0.99 & 0.98 & 0.99 \\


\hline


\end{tabular}

\caption{Recall Precision and F-Measure values for Partridge Topic Classifier}
\label{fig:fmeasure_table}

\end{figure}

Recall, Precision and F-Measure was calculated for each of the classes and in
each iteration of the three fold validation sequence. The results in Figure
\ref{fig:fmeasure_table} show good recall, precision and f-measure for all
three document types within Partridge. These results verify that case study is
the hardest document type to classify as discussed above.

\section{ User Testing } 

Raymond (1999) suggests that for developers of open source projects, ``...treating
your users as co-developers is your least-hassle route to rapid code
improvement and effective debugging\cite{raymond1999cathedral}." Partridge
hasn't been developed using an open methodology since, as a dissertation
project, it has been necessary to avoid complications around work ownership.
However, Raymond's notion that software should be released early and often and
tested on its user-base is an effective one and has been widely adopted by both
open and closed source developers\cite{linux2013}\cite{unity2013}.

A publicly-available version of Partridge that can be accessed at
(\url{http://farnsworth.papro.org.uk/}). This was made available as soon as iteration
1 was complete and advertised using several social media stream to attract as many users to Partridge as possible in the early stages of
the project in order to provide user testing. It was made clear to any early-adopting users that the system was still in development and they were
encouraged to submit bug reports if they experienced any problems whilst using
the site. This was a great success and a number of bugs and suggested
improvements were raised both via the GitHub repository and in conversation,
helping to improve Partridge and find some defects that had previously been
overlooked. For example, one user noticed that the full text search within
Partridge was not behaving as expected and returning all papers regardless of
their contents.  The relevant code was repaired and the bug marked as resolved
on GitHub\cite{softlybug2013}.


\subsection{Intrinsic Feedback} 

Intelligent information retrieval using machine learning techniques is a
complicated and multifaceted process. Partridge attempts to encode this complex set of
procedures into a usable and understandable interface designed for the use of
non-programmers. It was therefore important to understand how usable the
interface is for non-technical users.

Most users who commented on the system found that the query form made a lot of
sense and helped them to identify relevant papers very quickly. 

There were a few comments as to how the uploader system could be improved. For example, users
were confused by the preprocessor and wondered why their paper didn't appear in
the corpus as soon as they submitted it. The most popular solution to this was
to add an email field to the uploader so that people who submit papers to the
corpus are notified when their paper is incorporated. This also provided
feedback for users whose papers could not be added to the corpus due to errors
or rights ownership issues.

Another suggestion was to add a twitter bot to announce when new papers have
been added to the system. This would provide another way of issuing user
feedback when a paper is submitted and would also help to draw users to
Partridge - especially when high profile papers are successfully added to the
system.

\subsection{Extrinsic Feedback}

As well as finding out how Partridge performs as a system, it was interesting
to find out how Partridge blended into the information retrieval ecosystem around
it and how it could be effectively used day-to-day or in combination with any other
online tools.

One of the most important suggestions made was the addition of the Bookmarklet
tool. This had the effect of making paper submission to Partridge a lot more
convenient and intuitive. It also helped reduce confusion over paper
rights since the bookmarklet only works on sites which license
their papers under permissive licenses and allow data mining. 

\section{Summary}

The extensive range of testing methodologies used throughout the project meant
that Partridge has been developed into a mature, stable and highly usable
product. Unit testing has ensured that all procedural and scripted
functionality within Partridge is predictable and reliable. Machine Learning
Evaluation has ensured that an accurate and precise paper type classification
mechanism has been developed and User Testing has ensured that non-technical
users are able to use Partridge effectively.

Testing throughout the project has meant that any potential
issues with the system were identified very early on and resolved due to the
`open-source-esque' development lifecycle used during the project.  Testing
Partridge has helped in understanding the system's limitations and raised a
number of questions about its future development and ideas for additional
improvements. It has also provided a great deal of insight into how effective
Partridge is at its intended purposes, as well as its potential impact on
internet research.


\chapter{Critical Evaluation}
%%------------------------------------------------------
%  
%  Evaluation include for dissertation
%
%------------------------------------------------------

\subsection{ Achievements }

\subsection{ Limitations and Potential Improvements }

\subsection{ Further Research }


\pagebreak
\bibliographystyle{IEEEannot}
\bibliography{report}

\appendix

\appendix
\section{User Interface Designs}
\label{sec:ui_designs}

\subsection{ Partridge Query Interface}


Figure \ref{fig:ui_mockup} shows the user interface that will be utilised to
search and query Partridge's corpus. 

The query form shows two main sections: Filters and Keywords. 

Filters can be used to show a large set of papers where the user is unsure of
what to search for. Users will be able to look for papers by type and topic as
discussed above. The paper result filter has not been included in this diagram.
However, it is expected that a drop down menu for result type would be included
in the filters section of the form.

Keywords allows the user to look for a set of keywords that is specifically
within a CoreSC concept for a paper. For example, the reader may want to find
experiments that use a server farm to do lots of calculations. Therefore, they
would enter ``Server Farm" as their keywords and chose ``Method" from the paper
section. The user then adds this to the set of keyword queries in the table
below the text field with the add button.

Once a user has configured both filters and keywords to their preferences, they
click the search form to run the search itself.

Other notable features on this mockup include a list of most recent searches.
This helps the user to understand how to use Partridge and gives them some
suggestions for what they might search for. Similar listings are provided for
the most popular papers in the Partridge database and the most recent papers
added to the system.

\begin{figure}[!ht]
\includegraphics[width=\textwidth]{images/mockup_1.png}
\caption{User interface design for querying Partridge}
\label{fig:ui_mockup}
\end{figure}


\section{System Process Diagrams}
\label{sec:system_diagrams}

This section documents the system process diagrams that have been produced. The
first completed diagram shows how a paper will be added to Partridge and the
actions that will be carried out upon it.

\subsection{Adding a Paper to Partridge}

\begin{figure}[!ht]
\includegraphics[width=0.75\textwidth]{images/PaperAddedProcess.png}
\caption{The process of adding a new paper to Partridge}
\end{figure}

The system starts off by converting the document to XML from PDF if necessary
through the use of the PDFX tool provided by Manchester University. A sentence
splitting tool is then used to parse the resulting document and separate the
sentences.

Once the sentences have been split, Maria's SAPIENTA tool is used to annotate
each sentence to determine which of the core scientific concepts it covers.
This information is stored back in the XML file with the document.

After this, feature extraction is carried out upon the document to find useful
features for the following classification tasks:

\begin{enumerate}
\item Paper topic/subject -i.e. is the paper a biology paper or a computer science paper?
\item Type of paper - i.e. is the paper a case study, an experiment or a literature review?
\item Paper result - i.e. did the paper have a positive, negative or inconclusive result?
\end{enumerate}

These classifiers are then run (they could potentially be run in parallel if
processing power is available) to determine the paper's class for each
classifier respectively. The data gathered along with any paper metadata
captured such as title, author, date, institute etc are then stored in the
database.


\section{Email for CRFSuite}
\label{sec:crfemail}
\begin{verbatim}
Hi All,

I'm trying to build CRFSuite's python extension 
(http://www.chokkan.org/software/crfsuite/) on an Arch Linux 64-bit 
environment using Swig 2.0.8 (PCRE enabled).

When I run SWIG, I get the following output:

<<OUTPUT OMITTED FOR BREVITY>>

Full output available at http://sourceforge.net/mailarchive/message.php?msg_id=30024209

As far as I can see, all of these problems stem from this declaration in 
the SWIG input file (http://pastebin.com/9JipJ1C1):
%template(ItemSequence) std::vector<CRFSuite::Item>;

I've had a look around on google and on this mailing list and the only 
other "cannot copy typemap" errors I've encountered have been where 
people have excluded the std namespace in favour of a 'using' statement. 
As you can see, this example uses absolute class names so that isn't the 
issue here. I haven't had any luck contacting the software maintainer 
for CRFSuite yet either.

As you might expect, if I continue to compile the extension without 
heeding these errors, when I try to make use of the ItemSequence object 
I get the following python error:
TypeError: in method 'ItemSequence_append', argument 2 of type 
'std::vector< CRFSuite::Item >::value_type const &'

I'm going to keep trying to sort this out and if I find the solution 
independantly, I'll make sure to post a follow-up email. However, if 
anyone else has any ideas about what might be happening here, please let 
me know.

Thanks,

James Ravenscroft
AI & Robotics Undergraduate
Aberystwyth University
\end{verbatim}


\section{Project Timeline}
\label{sec:timeline}

This table shows the projected timeline for the Partridge project. These
calculations were carried out under the assumption of 2 full days of work per
week on the project and one iteration equating to one month (28 days). There
will be 7 iterations in total and after each, a new version of Partridge will
be made available (except for iteration zero which is a research only
iteration).


\begin{figure}[htp] 
 \vspace{ -1cm }
 \centering{\includegraphics[scale=0.60,page=1]{../timeline.pdf}}
  
\end{figure}

\begin{figure}[!htp] 
 \centering{\includegraphics[scale=0.60,page=2]{../timeline.pdf}}
 \vspace{-2cm}
\centering{\includegraphics[scale=0.60,page=3]{../timeline.pdf}}
\end{figure}

\pagebreak
\section{Project Wiki}
\label{sec:wiki}

All pages from the Wiki are shown below with the exceptions of those pages
which have already been mostly featured in the document (UI Mockups and
Progress Diagrams and the project timetable are not shown in this section).

\subsection{ Notes 03/10/2012 }
\subsection{Meeting Notes from 03/10/2012}

\subsubsection{General topics}

\begin{itemize}
\item
  Talked about using research papers instead of books :-
\item
  copyright issues - mainly resolved because of Gutenburg project
\item
  processing issues - too much information to process quickly and
  realistically
\item
  Maria has some software for identifying sections in papers - could be
  used to help locate specific features
\item
  There are quite a few sources for papers that could be used for the
  project:
\item
  http://arxiv.org/
\item
  http://plosone.org/
\item
  Maria's papers are in XML format
\item
  Formatting may or may not be an issue - PDFs are a nightmare to parse
\end{itemize}

\subsubsection{Things to consider}

\begin{itemize}
\item
  What sort of recommendations for different sources might there be and
  how would these recommendations be extracted?
\item
  Feature engineering - what sorts of features might there be to pick
  out in the papers?
\item
  I need to have a read through the sentiment analysis paper that Maria
  has been working on
\item
  I need to have a play with the software for identifying sections in
  papers.
\end{itemize}

\subsubsection{Misc concerns}

\begin{itemize}
\item
  Decided that our weekly meeting should be Wednesdays at 3pm
\item
  Investigate setting up source control and a wiki (that's been done as
  you can tell)
\end{itemize}


\subsection{ Notes 10/10/2012 }
\subsection{Meeting Summary 10/10/2012}

\subsubsection{From Maria's notes}

\begin{itemize}
\item
  James has gone through papers and thinks would be more doable to work
  with papers than Books
\item
  James has found the XML format of PlosOne and ART/CoreSC corpus
\item
  Domain needs to be determined
\item
  Maria to send paper on sentiment analysis of citations
\end{itemize}

\subsubsection{Current features and recommendations}

\begin{itemize}
\item
  Analysis of results section - positive or negative result
\item
  Potentially look at writing styles of authors using syntactic analysis
\item
  Isolating terms within sections of papers (i.e.~find me papers where
  methodology x was used)
\item
  finer grain analysis of CoreSC categories
\item
  nltk python toolkit. What other toolkits? Would jerboa be any good?
\item
  common features for sentiment analysis
\end{itemize}

\subsubsection{Basic system design}

I have started a basic system design although it is very basic at the
moment. [Image Omitted]

\subsubsection{Background - Similar Systems and how they work}

\begin{itemize}
\item
  Google Scholar
\item
  Mendeley
\item
  Citeulike
\item
  Tweeting or `liking' on facebook.
\end{itemize}

\subsubsection{Reading}

\begin{itemize}
\item
  TF-IDF
\item
  Take a brief look at journal of negative results
\item
  `Bigrams + trigrams' and syntactic pattern finding
\item
  Common features used in NLP
\item
  Sentiment analysis of citations in papers
\item
  `Bag of words'
\end{itemize}

\subsubsection{Coding/practical research}

\begin{itemize}
\item
  Play some more with SAPIENTA
\item
  Implement a (La)TeX to SciXML or pubmed command-line tool.
\end{itemize}


\subsection{ Notes 19/10/2012 }
\subsection{Notes from Meeting 18/10/2012}

\subsubsection{General Comments}

\begin{itemize}
\item
  I need to decide on a domain for the project. It may be that papers on
  a varied selection of topics would be a good start
\item
  Decided on an agile development cycle with a basic program that is
  improved in `iterations'.
\item
  As part of testing the software, it may be possible to get other final
  year students to trial the engine and use it to make recommendations
  for their reading.
\item
  I need to investigate ways of analysing papers for syntactic patterns.
  This may be indicative of:

  \begin{itemize}
  \item
    Different authors' writing styles
  \item
    Different types of paper -i.e.~psychology vs physics
  \end{itemize}
\end{itemize}

\subsubsection{TODO}

\begin{itemize}
\item
  Revisit PDF to XML conversion as conversion from PDF would make life a
  lot easier
\item
  Maria mentioned that she may have access to an HTML to XML program
  which would be good to look at.
\item
  Check out python Jerboa to see if it would be useful for the project.
\end{itemize}


\subsection{ Notes 24/10/2012 }
\subsection{Notes from meeting 24/10/2012}

Amanda was not present for this meeting.

\subsection{Discussed}

\begin{itemize}
\item
  We discussed the PDF conversion routine and decided that whilst
  important, it is crucial not to get too caught up in this.
\item
  Maria is going to send me a link to an existing PDF to XML converter
  that might provide the functionality with some tweaking - or at least
  help
\item
  We discussed the Python version of Sapienta which currently supports
  SciXML and simplified versions thereof.
\item
  Maria needs to send me a couple of data files for this to work
  properly.
\item
  Confirmed that Jerboa is a Java library (I was confused because I was
  expecting a python toolkit) and need to look at it properly now.
\item
  Discussed the actual functionality of the system and came up with a
  brief system outline, around which I can plan.
\end{itemize}

\subsection{TODO}

\begin{itemize}
\item
  Now that we know what the system is going to do (See System Outline), I need to look at features that
  will allow us to:
\item
  Pre-classify documents based on their topic, type, result etc
\item
  Determine document similarity based on user preference from their
  tagging etc. (I can make use of the ``Features in sentiment analysis''
  paper for this)
\item
  Run more tests on the PDF parser and potentially look at another
  solution if this takes up too much time.
\item
  Have a go with Python SAPIENTA once the data models are present
\end{itemize}


\subsection{ Notes 31/10/2012 }
\subsection{James' Meeting Notes}

\subsubsection{To Think about}

\begin{itemize}
\item
  Discussed imminent deadline for project progress report - 19/11/2012
\item
  Decided that having a report finished by 14/11/2012 would be a good
  report to give Amanda and Maria a chance to check over the report
  before handin.
\item
  Maria explained that it is possible to send a batch of papers to
  Sapienta rather than one at a time. She is going to provide
  information on how this can be achieved at some point during the week.
\item
  I should describe my working processes and why they are helpful.
\end{itemize}

\subsubsection{Todo}

\begin{itemize}
\item
  We discussed the merits of using the pdfx converter instead of my
  script - might have plug-and-play compatibility with Sapienta for a
  start - need to verify this.
\item
  I need to send my error message the CRFSuite compilation to Maria so
  that she can contact the maintainer.
\item
  I need to expand on the system outline providing more specific goals
  and timeframes - this will become my development plan.
\item
  I need to look at textpresso and see if it will help me with my
  comparison/evaluation
\end{itemize}


\subsection{ Notes 07/11/2012 }
\subsection{Notes from 7/11/2012}

\subsubsection{Discussed}

\begin{itemize}
\item
  What I'll need to have for my demo:

  \begin{itemize}
  \item
    Simple web interface
  \item
    Few basic classifiers - probably result of paper positive/negative
    and maybe paper type - proportions of coreSC concepts.
  \item
    Could show off PDF to XML converter script
  \end{itemize}
\end{itemize}

\subsubsection{To Consider}

\begin{itemize}
\item
  Is the disparity between bio paper CoreSC classification and other
  sorts of paper to do with Sapienta's model or does it demonstrate a
  feature - different subjects may have different proportions of each
  concept.
\item
  If I'm using distribution of concepts as an aggregate feature, I need
  to make sure the system does not fall over if a concept is missing.
\item
  Front end interface:
\item
  Need to find a new word for `section' Scientific Concept may be a bit
  too `scary', is there a better term for this?
\item
  If I allow users to upload papers, they must sign a digital disclaimer
  to say that they have the right to do so
\item
  There should be a ``report copyrighted material'' system
\end{itemize}

\subsubsection{To Do}

\begin{itemize}
\item
  Biggest task this week: Progress Report.

  \begin{itemize}
  \item
    Draft due 14/11/2012 for discussion on 15/11/2012
  \item
    Final deadline 19/11/2012
  \item
    Could provide wiki content as an appendix to the report
  \item
    Come up with a project timeline - actually consider dates
  \item
    Start with the web interface and work upwards.
  \end{itemize}
\item
  Send Maria the missing header file from crfsuite
\item
  Link system outline from wiki index
\item
  Maria said she'd look for a topic detection paper for me to peruse
\item
  I need to have another look at relevant features - struggling with
  this
\item
  Make a wiki list of features that could be used for classification
\end{itemize}




\end{document}
