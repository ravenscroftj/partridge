\section{Existing Methodologies}
Selecting a suitable development methodology for building Partridge is another
very important choice for the project.

Under the traditional `Waterfall' Software Development model, Requirements
Gathering, Analysis, Program Design, Coding, Testing and Operations were all
defined as formal phases in the development cycle. There is little flexibility
other than moving back up the waterfall to rectify mistakes after
testing\cite{Royce:1987:MDL:41765.41801}. This model was very focused on
paperwork and bureaucracy, trying to maintain a paper trail and manage risk
through accountability \emph{(Ibed)}. This approach to software development is
very heavyweight and slow and often produced software that did not match the
users' needs as a result \cite{Boehm1988}.

As an alternative to the heavyweight Waterfall approach, Beck \emph{et al.} came up
with the principle of the Agile Manifesto, favouring a lightweight, responsive
development model over the heavyweight slow waterfall
system\cite{beck2001agile}. Many of Beck's ideas focus around working in a team
of developers and prioriting communication between team members \emph{(Ibid.)}.
This is most prominent in the Extreme Programming (XP) method of software
development. Since Partridge is an individual project, XP was not really
applicable. However, some concepts like rapid prototyping/spike work and
iterative release cycles were used as part of the Partridge development
methodology.

\section{ Partridge's Development Methodology}

A custom methodology was used for keeping track of development within
Partridge.  All design and planning documentation have been written up and
placed on a wiki which is accessible and modifiable by the author and both
supervisors. This creates a paper trail for all tasks and also allows
collaboration between involved parties through the Internet.

Throughout the project, weekly meetings were held with both project
supervisors. The notes from the preceeding week were analysed and each task
discussed in depth. New tasks were then noted down along with any observations
that should be documented. These were then uploaded to the wiki the following
day or earlier. Each party present at the meeting adds their own observations
to the notes page. This page is then reviewed at the next meeting.

Partridge made use of an agile, `iterative' development process. A working
version of the system was released at the end of each month. Tasks to be
undertaken in each iteration were stored in GitHub's issue manager program.
Priority was given to those tasks that provided the most value for the least
time investment. Test users were also allowed to submit bug reports to the
tracker system, and these were given priority if they were serious. 


\subsection{ Work Timeline }

The tasks involved in Partridge have been carefully calculated and prioritised.
They were then added to the GitHub issue management system and a report
generated listing them in the order that they are expected to be
accomplished. This report can be seen in Appendix \ref{sec:timeline}. 


