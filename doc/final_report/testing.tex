%%------------------------------------------------------
%  
%  Testing include for dissertation
%
%------------------------------------------------------

\subsection{ Unit Testing }

\subsubsection{ Nose and Sniffer }

Nose for python is an extension of Python's built in Unit testing module that
is supposed to make setting up a testing environment and reusing test code
easier. Unit tests for Patridge's webserver backend and paper preprocessor were
written and maintained throughout the project. As each new feature was added to
the system, the whole suite of tests was executed to provide integration
coverage and make sure that new features did not break existing code. 

\subsubsection{ Selenium }

The frontend parts of Partridge that were written in Javascript and HTML5 also
had to be tested to verify that the query and upload forms behaved as expected.


\subsection{ User Testing } 

Once a basic user interface had been implemented, the system was opened for
user testing. The main Patridge instance
(\url{http://farnsworth.papro.org.uk/}) was advertised using several social
media streams and users encouraged to submit bug reports if they experienced
any problems whilst using the site.


\subsection{ Testing Learning Algorithms }

Patridge uses supervised machine learning algorithms to classify the type of
research papers added to its database. It was therefore necessary to test that
these algorithms provide accurate classification given the training and
features used in its model.  Unlike traditional procedural algorithms that can
be unit tested and in some cases formally proven to behave uniformly at
runtime\cite{filliatre2007formal}, supervised learning algorithms are imperfect
systems that build `working models' from known data. This makes the performance
of supervised learning systems much more difficult to assess. Russell and
Norvig(2010) suggest that ``a learning system is good if it produces hypotheses
that do a good job of predicting the classification of unseen
examples\cite{russell2010artificial}." It is suggested that when training a
supervised learning system, some pre-classified examples should not be used in
the training example and instead, the system is used to classify these
remaining examples, its decision compared to the actual, known class of
each\cite{alpaydin2004introduction}\cite{russell2010artificial}.

\subsubsection{ Partitioned Data Sets}

After a suitably large number of papers were collected and annotated, the
papers were partitioned into disjoint \emph{training} and \emph{testing}
sets. The decision tree that is used to classify a paper's "type"  
